\documentclass{article}

\pdfsuppresswarningpagegroup=1
\hbadness=99999
\usepackage[a4paper, margin=2cm]{geometry}
\usepackage{amsmath}
\usepackage{array}
\usepackage{hyperref}
\usepackage{pdflscape}
\usepackage{longtable}
\usepackage{booktabs}
\usepackage{tikz}
\usepackage{svg}
\usepackage{placeins}
\usepackage{tabularx}
\usepackage{ltablex}
\usepackage{eso-pic}
\usepackage{tablefootnote}
\usepackage{amssymb}
\keepXColumns
\newcolumntype{L}{>{\raggedright}X}
\usetikzlibrary{positioning}
\tikzset{
  script/.style={draw, circle},
  utxo/.style={draw, rectangle, align=center}
}

\AddToShipoutPictureBG{%
\begin{tikzpicture}[remember picture, overlay]
\node[opacity=.5, inner sep=0pt]
  at(current page.center){\includegraphics{
  images/background.png}};
\end{tikzpicture}%
}

% Avoid hyphenation
\tolerance=1
\emergencystretch=\maxdimen
\hyphenpenalty=10000
\hbadness=10000

% Style hyperlinks
\hypersetup{%
  colorlinks=true,
  linkcolor=blue,
  linkbordercolor=red,
  pdfborder={0 0 0}
}

% Cover page image
\AddToShipoutPicture*{%
\put(0,0){%
\parbox[b][\paperheight]{\paperwidth}{%
\vfill
\centering
\includegraphics[width=\paperwidth,height=\paperheight]{
images/coverpage.jpg}%
\vfill
}}}

\renewcommand*\contentsname{Table of Contents}

\begin{document}
\begin{sloppypar}

\begin{titlepage}
\sffamily\selectfont
\centering
\vspace*{7cm}
{\includesvg[width=10cm]{images/logo.svg}}\\[5\baselineskip]
\textcolor{white}{
{\Huge Synthetic Assets on Cardano}\\[2\baselineskip]
{\Large Indigo Laboratories, Inc.}\\[0.5\baselineskip]
{\large info@indigo-labs.io}\\[3\baselineskip]
{\large November 2022, v1.0}
}
\end{titlepage}

\tableofcontents

\hypertarget{motivation}{%
\section{Motivation}\label{motivation}}

For most of the world's population, important financial tools are
inaccessible. This is exemplified by the fact that two people can share
the same education, perform the same work, and put in the same amount of
effort, yet not have the same development possibilities. One could have
access to a share of the global economy's growth, while the other may be
left out.

It's brutal and unfair. Until now, borders have limited human
development. With the advent of blockchain technology, we are amid a
global switch in the financial foundation we use as a society to trade
and transact. At the forefront of this transformation, we present a new
solution to equalize the playing field by bringing the world's assets to
the blockchain. This solution allows anyone to access and participate in
new financial markets and take control of their own financial destiny,
paving the way for a new mantra: Tokenize Everything.

\hypertarget{introduction}{%
\section{Introduction}\label{introduction}}

This document (the ``Indigo paper'') presents the Indigo Protocol (the
``protocol'' or ``Indigo''), a synthetic assets protocol built for
Cardano\footnote{\href{https://www.coinbase.com/learn/crypto-basics/what-is-cardano}{Cardano}
  is a public blockchain that supports smart contracts and custom tokens
  utilizing an
  \href{https://docs.cardano.org/plutus/eutxo-explainer/}{eUTXO}
  architecture, an extension of
  \href{https://unchained.com/blog/what-is-a-utxo-bitcoin/}{UTXO}.}.
Combining the benefits of a white paper\footnote{A
  \href{https://cointelegraph.com/funding-for-beginners/what-is-a-white-paper-a-beginners-guide-on-how-to-write-and-format-one}{white
  paper} is a marketing tool typically used to attract investors.} and a
yellow paper\footnote{A
  \href{https://wikicryptocoins.com/currency/Yellow_Paper}{yellow paper}
  typically contains complete specification details.}, the Indigo paper
provides both a high-level and detailed protocol specification for
educating the Indigo community. The Indigo paper serves as the basis to
introduce Indigo and kickstart complete community management.

Indigo has been committed to a \protect\hyperlink{fair-launch}{Fair
Launch} to bootstrap the protocol from the ground up. As part of this
initiative, no minting, pre-sale, or distribution of tokens related to
the protocol have been undertaken. This ensures that starting from
launch, Indigo will be community managed.

\hypertarget{synthetic-assets}{%
\subsection{Synthetic Assets}\label{synthetic-assets}}

Indigo creates synthetic assets which are known in the protocol as
iAssets (i.e., ``Indigo Assets''). iAssets are cryptocurrency assets
that derive their prices from tracked assets. Prices of iAssets are
influenced via protocol rules with the intention of matching the prices
of the tracked assets. One example of an iAsset is iBTC, representing a
synthetic version of Bitcoin (BTC); it is designed to mimic the price
action of BTC -- an asset that lives in separate ecosystem than Indigo.

\hypertarget{indigo-protocol}{%
\subsection{Indigo Protocol}\label{indigo-protocol}}

Indigo is a decentralized synthetics protocol for on-chain exposure to
assets with publicly verifiable prices. Using Cardano Plutus\footnote{\href{https://developers.cardano.org/docs/smart-contracts/plutus/}{Plutus}
  is a smart contract platform for Cardano.} smart contracts, the
protocol enables the creation of iAssets. Prices of iAssets are soft
pegged\footnote{A
  \href{https://coinmarketcap.com/alexandria/glossary/soft-peg}{soft
  peg} is a strategy of maintaining the value of an asset against
  another asset by utilizing an exchange rate mechanism.} to external
tracked assets; iAssets are overcollateralized in the form of a
decentralized Collateralized Debt Position (``CDP''). The protocol
enforces liquidations to ensure iAssets always maintain
overcollateralization, meaning the value of the collateral in the CDP
exceeds the intended value of the iAsset. In the event of a CDP becoming
undercollateralized, a liquidation reestablishes overcollateralization
by confiscating the collateral of the undercollateralized CDP and
replacing it with another user's overcollateralized CDP.

While minting an iAsset requires opening a CDP, after iAssets are minted
they are freely exchangeable. Anyone with a Cardano wallet can send or
receive iAssets, regardless of whether they have an open CDP.

\hypertarget{benefits-of-iassets}{%
\subsection{Benefits of iAssets}\label{benefits-of-iassets}}

Users can gain some benefits of owning an asset without being required
to obtain or own the asset themselves. This can be useful in cases where
assets are difficult for a user to obtain or for assets that live
elsewhere yet users desire to utilize on the Cardano blockchain.

iAssets can be used as building pieces to be included in a wider
financial strategy. This could include being part of derivative
contracts or constructing a widely diversified portfolio in one easy to
use system. Users can make trades without requiring the underlying
supply. For example, more iAsset could exist than total supply of the
real asset, allowing for leveraged trades that wouldn't be possible to
be settled using the real underlying assets.

iAssets have the following properties:

\begin{itemize}
\item
  Tracking different type of assets and statistics; which allows the
  creation of many new asset classes for emerging industries.
\item
  No custodians; iAsset creation is fully decentralized.
\item
  Low barrier to entry; anyone with cryptocurrency can use Indigo to
  mint new synthetic assets or buy and trade them on the open market.
\item
  Composability; iAssets can be used as a lego block, enabling their
  integration into a larger financial ecosystem.
\end{itemize}

\begin{tabularx}{\linewidth}{l|l}
\caption{Examples of possible iAssets}
\tabularnewline
\toprule
\textbf{Name} & \textbf{Description}
\tabularnewline
\midrule
\endhead
\textbf{iBTC} & Tracks the price of BTC on the Bitcoin
blockchain
\tabularnewline
\midrule
\textbf{iETH} & Tracks the price of ETH on the Ethereum
blockchain
\tabularnewline
\midrule
\textbf{iUSD} & Tracks the price of dollar-denominated stablecoins on
any blockchain
\tabularnewline
\midrule
\textbf{iCPI} & Tracks the change of the Consumer Price Index over
time
\tabularnewline
\bottomrule
\end{tabularx}

\footnotetext{The
\href{https://www.bls.gov/cpi/}{Consumer Price Index} is a measure
of the average change over time in the prices of goods and
services.}

Generally, an iAsset names begin with the letter ``i,'' followed by the
name of the tracked asset.

\hypertarget{obtaining-iassets}{%
\subsubsection{Obtaining iAssets}\label{obtaining-iassets}}

There are two ways to obtain iAssets:

\begin{itemize}
\item
  \textbf{Buying iAssets --} Users directly purchase iAssets via an
  exchange (centralized or decentralized), thus gaining exposure without
  having to request any loan.
\item
  \textbf{Minting iAssets --} Users make interest-free,
  overcollateralized loans against their cryptocurrency assets.
\end{itemize}

Users can buy iAssets from any exchange that has available supply. After
buying an iAsset, the user gains full control of the iAsset and can reap
benefits from price possible appreciation. Users can be assured that
iAssets maintain their intended pegged prices due to Indigo's
liquidation process.

The second way for users to obtain iAssets is by minting them within the
Indigo Web App by depositing collateral and creating a loan.

\hypertarget{collateralized-debt-positions}{%
\subsection{Collateralized Debt
Positions}\label{collateralized-debt-positions}}

Every iAsset is backed by collateral held in a Collateralized Debt
Position (a ``CDP''). A CDP is created by a user locking collateral (a
minimum of 10 ADA) into Indigo to mint a new iAsset. An iAsset is
borrowed against the \textbf{collateral}, creating a \textbf{debt}, and
this \textbf{position} is watched by liquidators to ensure
overcollateralization.

The value of the collateral in a CDP should always exceed a
governance-based Minimum Collateral Ratio (a ``MCR''). Each iAsset type
has its own MCR. Both the value of the collateral and iAsset price can
fluctuate over time, potentially causing a CDP to become
undercollateralized. A CDP is considered undercollateralized when its
collateral ratio (the ``CR'') falls below the iAsset's MCR. The CR is
the ratio of the collateral value relative to the minted iAsset value,
and can be calculated using the formula:

\[c = \frac{a}{md}\]

or:

\[c = \frac{ab}{mp}\]

Where:

\begin{itemize}
\item
  \(c\) is the CR used to determine solvency
\item
  \(a\) is the amount of ADA locked in the CDP
\item
  \(b\) is the dollar-denominated price of ADA
\item
  \(p\) is the dollar-denominated price of the iAsset's tracked asset
\item
  \(d\) is the ADA-denominated price of the iAsset's tracked asset
\item
  \(m\) is the amount of iAsset minted from the CDP
\end{itemize}

When CR drops below MCR, the CDP is considered insolvent and eligible
for being frozen, which can then lead to
\protect\hyperlink{stability-pools}{liquidation} to ensure the
reestablishment of solvency.

\hypertarget{cdp-and-iasset-example}{%
\subsubsection{CDP and iAsset Example}\label{cdp-and-iasset-example}}

As an example, assume Violet wants to mint 100 iDOT (\(m\)). DOT is
trading for \$15 (\(p\)). Violet has 2,000 ADA (\(a\)) she's willing to
use as collateral to borrow iDOT. ADA is trading for \$1.28 (\(b\)).

Violet deposits 2,000 ADA into Indigo to mint 100 iDOT. A CDP is created
consisting of 2,000 ADA. Violet now owns 100 iDOT and owes 100 iDOT to
Indigo. Violet can still earn
\protect\hyperlink{cdp-liquid-staking}{staking rewards} from her 2,000
ADA, but cannot transfer it because it now is used as collateral. To
regain control of her ADA, Violet must return 100 iDOT.

Violet's CR is \textasciitilde171\%:

\[c = \frac{ab}{mp} \therefore \frac{2000 \times 1.28}{100 \times 15} = \  \sim 1.71\]

As the price of either DOT or ADA changes, CR changes too. When CR drops
below the iDOT's MCR, Violet's CDP is subject to liquidation.

If the price of ADA increases to \$1.40 (\(b\)) and DOT increases to
\$19 (\(p\)), then Violet's CR drops to \textasciitilde147\%:

\[\frac{2000 \times 1.4}{100 \times 19} = \sim 1.47\]

If the iDOT MCR was 150\%, Violet's CDP could be liquidated. Upon
liquidation, Violet would lose her 2,000 ADA collateral deposit. Violet
could still have her 100 iDOT, worth \$1,900 (\$19 x 100). The 2,000 ADA
she lost would be worth \$2,800 (\$1.40 x 2000). Therefore, Violet could
have lost \$900 of value (\$2,800 - \$1,900).

To have prevented liquidation, Violet needed to either add more ADA into
her CDP to increase its CR, or close the CDP by returning the 100 iDOT
she borrowed.

\hypertarget{cdp-actions-and-states}{%
\subsubsection{CDP Actions and States}\label{cdp-actions-and-states}}

Several actions can be taken against a CDP by users of the protocol:

\begin{itemize}
\item
  \textbf{Open Position --} Creates a CDP by a user depositing a minimum
  of 10 ADA as collateral, and associates it with an iAsset type that
  can be minted. The user who creates the CDP becomes the CDP's owner.
\item
  \textbf{Deposit Collateral --} An owner can increase CR by depositing
  more collateral.
\item
  \textbf{Withdraw Collateral --} An owner can lower CR by withdrawing
  part or all the collateral. Collateral cannot be withdrawn if it
  brings CR below the iAsset's MCR. If a CDP has no debt (i.e., no
  minted iAsset) and all collateral is withdrawn, then the CDP is
  closed.
\item
  \textbf{Borrow iAsset --} An owner can lower CR by minting an iAsset.
  This increases the amount of debt against the CDP. More iAsset cannot
  be minted if it brings CR below the iAsset's MCR.
\item
  \textbf{Repay Debt --} An owner can increase the CR by repaying debt
  in the form of iAsset. When the debt is repaid, the iAsset is burned
  (i.e., destroyed). More iAsset cannot be burned than debt owed by the
  CDP.
\item
  \textbf{Freeze --} If CR is below the iAsset's MCR, any user can
  submit a transaction for that CDP to be frozen. Upon freezing, a CDP
  is no longer usable or interactable by its former owner. The former
  owner loses all access and rights to the CDP.
\item
  \textbf{Liquidate --} If a CDP is frozen, any user can submit a
  transaction for that CDP to be liquidated. Upon liquidation, CDP debt
  is repaid by withdrawing iAsset from a
  \protect\hyperlink{stability-pools}{Stability Pool}. As debt is
  repaid, collateral is withdrawn from the CDP. If all debt is repaid,
  then all collateral is withdrawn, and the CDP is closed.
\item
  \textbf{Merge --} If two or more CDPs are frozen, any user can submit
  a transaction for them to be merged into one CDP. Upon merging, all
  but one of the CDPs requested to be merged are closed, and their debt
  and collateral are transferred into a single CDP.
\end{itemize}

A CDP has the following states:

\begin{itemize}
\item
  \textbf{Open --} A CDP that is fully collateralized, with its CR value
  above the iAsset's MCR. Open CDPs remain fully usable by their owners.
\item
  \textbf{Insolvent --} A CDP that is undercollateralized, with its CR
  value below the iAsset's MCR. Insolvent CDPs remain fully usable by
  their owners but eligible to be frozen by any user.
\item
  \textbf{Frozen --} A CDP that has been confiscated by the protocol and
  no longer has an owner. A CDP becomes frozen after a user successfully
  submits a request against an insolvent CDP. Frozen CDPs cannot be used
  by their former owners.
\item
  \textbf{Closed --} A CDP whose CR value is zero, no longer having any
  collateral or debt. A CDP is closed after all its debt is repaid and
  its collateral is withdrawn.
\end{itemize}

\hypertarget{cdp-liquid-staking}{%
\subsubsection{CDP Liquid Staking}\label{cdp-liquid-staking}}

Indigo supports liquid staking of ADA collateral within CDPs, allowing
users to continue earning ADA rewards from the Cardano network on top of
utilizing the benefits of iAsset minting. This improves capital
efficiency and doubles reward capabilities -- rewards earned from
Cardano, and rewards earned from Indigo. Liquid staking is a unique
capability offered by Indigo and will help attract liquidity from
outside of the Cardano ecosystem to encourage more participation,
bringing iAssets to a wider audience.

To use liquid staking, users must first have their Cardano wallet staked
to their preferred stake pool\footnote{Indigo supports any
  \href{https://cardano.org/stake-pool-delegation/}{Cardano stake pool}.
  A stake pool is a Cardano network node that forms the basis for
  consensus on the blockchain. Users can
  \href{https://www.adastrong.com/staking/}{delegate their ADA} to stake
  pools to earn ADA rewards from the Cardano network.}. The Indigo Web
App automatically attaches the user's staking key when creating a CDP.
All ADA deposited into that CDP will continue to earn staking rewards
from the users's chosen stake pool, accruing in the user's wallet.

If the user delegates their wallet to a new stake pool after creating
the CDP, the CDP will automatically earn rewards from the new stake
pool.

\filbreak

\hypertarget{indy}{%
\subsection{INDY}\label{indy}}

The Indigo DAO Token (``INDY'') is a Cardano native asset that can be
owned, held, or transferred by any user. INDY serves as Indigo's utility
token, with one of its key purposes being to allow on-chain voting on
DAO proposals (a ``proposal'')\footnote{The usage of INDY and the
  process of voting is described further in the
  \href{https://github.com/IndigoProtocol/indigo-dao}{Indigo DAO
  Constitution and Voting Procedures}.}. The total supply of INDY is 35M
with a 6 decimal precision. INDY's monetary policy disallows future
minting and burning, therefore making the total supply constant and
unchanging. Indigo is undergoing a \protect\hyperlink{fair-launch}{Fair
Launch}, therefore there has been no pre-sale nor private distribution
to investors prior to launch.

INDY will be distributed every Cardano epoch (five days), over a period
of five years. There will be three distribution schedules for the
community:

\begin{itemize}
\item
  \textbf{Governance Distribution --} Users who opt to stake their INDY
  into Indigo and participate in \protect\hyperlink{governance}{DAO
  Governance} by voting on proposals will be eligible for INDY rewards
  proportionally to their pro-rata share of staked INDY.
\item
  \textbf{Stability Pool Distribution --} Users who stake their iAssets
  in \protect\hyperlink{stability-pools}{Stability Pools} to ensure the
  protocol's solvency will be eligible for INDY rewards proportionally
  to their pro-rata share of staked iAssets.
\item
  \textbf{Liquidity Distribution --} Users who provide
  \protect\hyperlink{liquidity-staking-rewards}{liquidity} in DEXes and
  stake their LP tokens in Indigo will be eligible for INDY rewards
  proportionally to their pro-rata share of staked LP tokens.
\end{itemize}

\hypertarget{indy-distribution}{%
\begin{figure}[htbp]
\centering
\includesvg[width=17cm,inkscapelatex=false]{
images/tokenomics/distribution.svg}
\caption{Distribution of INDY over five years}
\end{figure}}

\filbreak

\hypertarget{fair-launch}{%
\subsubsection{Fair Launch}\label{fair-launch}}

Indigo has approached its tokenomics and launch from a new perspective,
with a focus on gaining community trust first, allowing the protocol to
be built with a vision of fairness. Early supporters of Indigo will be
among the first receivers of INDY for use within the protocol. INDY will
be distributed predominantly to users of the project, rather than
investors or special insiders.

After being in development for almost two years without investor
funding, the initial Core Contributors of Indigo who have built the
codebase -- and will continue to improve, optimize, and develop new
features -- will receive tokens vested over two years beginning the day
of mainnet launch. This is part of a Fair Launch to compensate work that
up until now has not been compensated.

Indigo has not minted, sold, allocated, distributed, or promised any
tokens to third parties. The purpose of INDY is to be used within the
protocol; until the launch of mainnet, there is or has been no use for
INDY to be distributed or sold. Indigo's Fair Launch has helped
alleviate community concerns over rug-pulling or the team not delivering
a useful and highly functional product. No purchasing of tokens will be
possible until the community has an opportunity to see and use Indigo
for themselves.

Indigo's Fair Launch is a novel approach to bootstrapping liquidity,
allowing the Indigo community to become highly collaborative, driven,
and vibrant. This is evidenced by the Indigo DAO Kickstart -- an effort
to decentralize the launch of Indigo -- which has received wide praise.
This approach bolsters Indigo's core tenet of decentralization, making
the launch itself a decentralized decision involving possibly thousands
of individuals from around the world. Indigo will be governed by the
community immediately upon launch. There will be no barriers for use.
Anyone, regardless of traits, will be able to gain benefit from Indigo's
iAssets. Indigo has established a new framework to allow for
community-led projects to come to life, which will be used to generate
INDY in as fair of a manner as possible.

\hypertarget{token-generation-event}{%
\subsubsection{Token Generation Event}\label{token-generation-event}}

Indigo's Token Generation Event (the ``TGE'') will occur upon the
beginning of, and at no point prior to, deployment of the Indigo
Protocol to mainnet (which is currently anticipated to be November
20\textsuperscript{th}, 2022). Upon minting of INDY, the Initial Token
Distribution (the ``ITD'') will be as follows:

\begin{itemize}
\item
  350,000 INDY to two or three DEXs approved by the Indigo community
\item
  350,000 INDY to participants within the Indigo community
\item
  21,000,000 INDY to one or more wallets (administrated by Indigo
  Laboratories, Inc. at the direction of the Indigo Foundation on behalf
  of the \protect\hyperlink{governance}{Indigo DAO}) to be used for the
  sole purpose of community rewards distributions
  (\protect\hyperlink{stability-pool-staking-rewards}{Stability Pools},
  \protect\hyperlink{liquidity-staking-rewards}{Liquidity}, and
  \protect\hyperlink{staking}{Governance})
\item
  4,550,000 INDY to the \protect\hyperlink{indigo-dao-treasury}{DAO
  Treasury Reserve}
\item
  8,750,000 INDY will be allocated to Indigo Laboratories, Inc. for past
  and expected future building, administering, and further developing
  the protocol, with 7,875,000 being distributed to team members under a
  two-year monthly vesting schedule
\end{itemize}

\hypertarget{indy-piechart}{%
\begin{figure}[htbp]
\centering
\includesvg[width=10cm,inkscapelatex=false]{
images/tokenomics/piechart.svg}
\caption{Allocation of INDY}
\end{figure}}

At launch, the circulating supply of INDY\footnote{A full detailed
  spreadsheet of the distribution of INDY with specific dates and
  allocations can be found in the open source
  \href{https://github.com/IndigoProtocol/indy-tokenomics}{indy-tokenomics}
  project.} will be 1,903,125; 350,000 of which being allocated to
Cardano DEXs via an Initial Liquidity Event.

\hypertarget{initial-liquidity-event}{%
\subsubsection{Initial Liquidity Event}\label{initial-liquidity-event}}

Indigo's Initial Liquidity Event (the ``ILE'') will distribute and make
INDY publicly available. The ILE will consist of three phases in
conjunction with the launch of Indigo:

\begin{enumerate}
\item
  Airdrop
\item
  Liquidity Bootstrapping Event (the ``LBE'')
\item
  Liquidity Pool Creation
\end{enumerate}

\hypertarget{indigo-airdrop}{%
\paragraph{Indigo Airdrop}\label{indigo-airdrop}}

Indigo's airdrop will distribute 350,000 INDY to participants within the
Indigo community. The airdrop will consist of two phases:

\begin{enumerate}
\item
  Distribution to early participants of the Indigo community
\item
  Distribution to stakers supporting the decentralization of Cardano and
  Indigo
\end{enumerate}

Each phase will be distributed 175,000 INDY. Cardano wallet addresses
have been collected by Indigo Laboratories, Inc. (the ``Labs'') and will
be forwarded to Vending Machine.\footnote{\href{https://vm.adaseal.eu/about}{Vending
  Machine} is a Cardano token distribution system.} The Labs will send
350,000 INDY to Vending Machine, who will subsequently distribute INDY
to qualified recipients via the Indigo Web App.

To redeem airdropped INDY, qualified users will need to connect their
wallet to the Indigo Web App and follow the in-app instructions to
withdraw INDY into their wallets. Users will be able to determine
whether they qualify for the airdrop upon connecting their wallets and
navigating to the appropriate reward page. Users will have until March
31\textsuperscript{st} 2023 to withdraw their INDY rewards into their
wallets. Any INDY not withdrawn by this time will not be eligible to be
withdrawn by users and instead will be subject to redistribution by the
Labs.

Members or affiliates of the Labs or Indigo Foundation make no promises
on the distribution of tokens. No action or series of actions guarantees
a user to receive INDY.

\hypertarget{airdrop-1-distribution-to-early-participants}{%
\paragraph{Airdrop 1: Distribution to Early
Participants}\label{airdrop-1-distribution-to-early-participants}}

Qualified participants for Airdrop 1 fit into either one of two
categories:

\begin{enumerate}
\item
  Participants who showed their interest by successfully completing each
  of the processes, which were:

  \begin{enumerate}
  \item
    Participate in Indigo's
    \href{https://forum.indigoprotocol.io/t/indigo-initial-token-distribution-vote/1399}{first
    temperature check} in the Indigo Forum
  \item
    Connect their Indigo Forum account with their Discord account
  \item
    Complete the \href{https://quiz.indigoprotocol.io/}{Indigo Quiz} to
    become an Indigo Guru
  \end{enumerate}
\item
  Participants who aided the Indigo community, as identified by the
  Labs' team
\end{enumerate}

172,751.924982 INDY is to be distributed to wallets that fit into the
first category, and 2,248.07304 INDY is to be distributed to wallets
that fit into the second category. A total of 3,458 wallets qualified
for the first category, and 30 wallets qualified for the second
category.

Addresses deemed to be suspicious or fraudulent were removed from the
first category.

\hypertarget{airdrop-2-distribution-to-decentralization-stakers}{%
\paragraph{Airdrop 2: Distribution to Decentralization
Stakers}\label{airdrop-2-distribution-to-decentralization-stakers}}

175,000 INDY will be distributed as a reward to users who helped boost
decentralization of the Cardano network by staking with a member of the
\href{https://singlepoolalliance.net/}{Cardano Single Pool Alliance}
(CSPA). To have qualified for receiving this reward, a user had to have
been staking a minimum of 10 ADA in one of 357 pools on November
6\textsuperscript{th}, 2022. A total of 79,679 wallets qualified to be
eligible to withdraw rewards. Each user who connects a qualified wallet
to the Indigo Web App will be eligible for a one-time withdrawal of 5
INDY on a first come first serve basis.

\hypertarget{indigo-liquidity-bootstrapping-event-and-liquidity-pool-creation}{%
\paragraph{Indigo Liquidity Bootstrapping Event and Liquidity Pool
Creation}\label{indigo-liquidity-bootstrapping-event-and-liquidity-pool-creation}}

In partnership with \href{https://minswap.org/}{Minswap}, Indigo will
begin a Liquidity Bootstrapping Event (the ``LBE'') on November
14\textsuperscript{th} 2022\footnote{More information about Indigo's LBE
  will be available on
  \href{https://indigoprotocol1.medium.com/}{Indigo's Medium}.}. The
goal of the LBE is to use a decentralized and transparent process to
discover a fair price for INDY. After the LBE starts, users can deposit
ADA into the Minswap Launch Bowl. Deposited ADA will be used to create
INDY/ADA Liquidity Pools (a ``LP'').

The Minswap LP will consist of 75\% of deposited ADA in the LBE paired
with 262,500 INDY. Depending on slippage analysis at the time of the LBE
end date of November 20\textsuperscript{th} 2022, 25\% of deposited ADA
in the LBE paired with 87,500 INDY will be used to create LPs on either
one or two DEXs approved by the Indigo community.

\hypertarget{stability-pools}{%
\subsection{Stability Pools}\label{stability-pools}}

A Stability Pool (a ``SP'') helps maintain iAsset solvency by acting as
the source of liquidity to repay debt from liquidated CDPs, thus
intending all minted iAsset supply to remain overcollateralized.

Every supported iAsset has its own SP (e.g., iBTC SP, iETH SP). A user
can deposit corresponding iAsset into a SP to become a SP staker (a ``SP
staker''). SP stakers provide stability to the protocol by offering
their iAssets to be used for liquidations.

SP Liquidation (``SPL'') is the process of utilizing a SP to liquidate a
CDP, where iAsset deposited in a SP are burned to repay the debt of an
undercollateralized CDP. In exchange, SP stakers earn a share of the
collateral that was confiscated from liquidated CDPs. When CR falls
below the iAsset MCR, the CDP is considered insolvent and subject to
liquidation, which amounts to canceling the debt where:

\begin{enumerate}
\item
  the same amount of iAsset debited by the CDP is burned from the
  corresponding SP; and
\item
  the collateral from the CDP is proportionally distributed to SP
  stakers.
\end{enumerate}

As CDPs become liquidated, SP stakers lose a pro-rata share of their
iAsset deposits while gaining a pro-rata share of the liquidated
collateral. An incentive for SP stakers to participate in SPL is the
possibility of earning net gains from liquidations. Under normal
circumstances, the value of the collateral earned may be greater than
the value of the canceled debt, because a liquidated CDP is likely to
have a CR value above 100\% (the value of the iAsset).

SPL first requires that CDPs are frozen. Each liquidation request of a
CDP is executed against its iAsset's associated single SP. Optionally,
users can make requests for CDPs to be merged. As illustrated in the
\protect\hyperlink{cdp-merge-figure}{CDP merge figure}, three CDPs could
be merged into a single CDP. The resulting merged CDP can then be
liquidated against the SP. While only a single liquidation can occur per
SP at once, multiple CDPs can be merged in parallel. Merging CDPs
effectively enables multiple frozen CDPs to be liquidated
simultaneously.

\hypertarget{cdp-merge-figure}{%
\begin{figure}[htbp]
\centering
\includesvg[width=10cm,inkscapelatex=false]{images/merge-cdp.svg}
\caption{Three CDPs being merged into one}
\label{cdp-merge-figure}
\end{figure}}

Indigo allows for both full and partial liquidations. A full
liquidation, as illustrated in the \protect\hyperlink{spl-figure}{SPL
figure}, repays all debt of a CDP and closes the CDP. A partial
liquidation, as illustrated in the
\protect\hyperlink{spl-figure}{partial SPL figure}, repays some debt of
a CDP and keeps the remaining position frozen. If a CDP debt is higher
than the entire amount of iAssets in the related SP, the protocol
attempts to cancel as much debt as possible with the iAsset supply
available. Any remaining non-liquidated collateral and debt of the CDP
remains frozen until more iAsset is deposited into the SP and another
liquidation is initiated.

\hypertarget{spl-figure}{%
\begin{figure}[htbp]
\centering
\includesvg[width=7cm,inkscapelatex=false]{images/spl.svg}
\caption{Illustration of a full liquidation where sufficient funds are
present in the SP}
\label{spl-figure}
\end{figure}}

\hypertarget{partial-spl-figure}{%
\begin{figure}[htbp]
\centering
\includesvg[width=7cm,inkscapelatex=false]{images/partial-spl.svg}
\caption{Illustration of a partial liquidation where there are
insufficient funds present in the SP}
\label{partial-spl-figure}
\end{figure}}

\filbreak

Values for the liquidated CDP and associated SP can be calculated using:

\[w = a - min\left\{ a,b \right\}\]
\[x = d - min\left\{ a,e \right\}\frac{d}{a}\]
\[y = e - min\left\{ a,e \right\}\]
\[z = g + min\left\{ a,e \right\}\frac{d}{a}\]

Where:

\begin{itemize}
\item
  \(w\) is the updated debt of the CDP after liquidation
\item
  \(x\) is the updated amount of collateral in the CDP after liquidation
\item
  \(y\) is the updated amount of iAsset in the SP after liquidation
\item
  \(z\) is the updated amount of ADA rewarded to the SP after
  liquidation
\item
  \(a\) is the amount of debt of the CDP before liquidation
\item
  \(b\) is the amount of iAsset in the SP
\item
  \(d\) is the amount of collateral in the CDP before liquidation
\item
  \(e\) is the amount of iAsset in the SP before liquidation
\item
  \(g\) is the amount of ADA rewarded to the SP before liquidation
\end{itemize}

\hypertarget{stability-pool-staking-fees}{%
\subsubsection{Stability Pool Staking
Fees}\label{stability-pool-staking-fees}}

Users can stake and unstake iAssets from SPs at any time. To stake
iAsset, a user needs to create a SP account by depositing 7 ADA and the
amount of iAsset they desire to stake. 2 ADA is returnable to the user
upon closing the SP account, which involves withdrawing all their iAsset
and earned rewards. 5 ADA is taken as a fee. Users pay a 1 ADA fee for
each new iAsset deposit into their SP account.

SP fees are collected and distributed to all SP stakers as part of
liquidation rewards.

\hypertarget{stability-pool-liquidation-rewards}{%
\subsubsection{Stability Pool Liquidation
Rewards}\label{stability-pool-liquidation-rewards}}

As liquidations occur, SP stakers lose a pro-rata share of iAsset
deposits and gain a pro-rata share of ADA rewards. A SP ``product
constant'' maintains mathematical state of liquidations occurred. When a
SP is first created, its product constant is set to one. Upon
liquidation, the product constant can be calculated using the formula:

\[c=a\left ( 1 - \frac{b}{d} \right )\]

Where:

\begin{itemize}
\item
  \(c\) is the new product constant
\item
  \(a\) is the current product constant
\item
  \(b\) is the amount of iAsset debited from the SP for the liquidation
\item
  \(d\) is the total amount of iAsset in the SP
\end{itemize}

A SP ``compounded constant'' maintains the mathematical state of rewards
earned from liquidations relative to the product constant. When a SP is
first created, its compounded constant is zero. Upon liquidation, the
compounded constant can be calculated using the formula:

\[r = a + \frac{bc}{d}\]

Where:

\begin{itemize}
\item
  \(r\) is the new compounded constant
\item
  \(a\) is the current compounded constant
\item
  \(b\) is the amount of ADA earned during the liquidation
\item
  \(c\) is the product constant before the liquidation
\item
  \(d\) is the total amount of iAsset in the SP before the liquidation
\end{itemize}

When an action is taken against a SP, such as a deposit of an iAsset or
a liquidation, its state is updated. The SP state data structure --
represented in the \protect\hyperlink{stability-pool-state}{SP state
table} -- is stored within the UTXO of the SP; ``iAsset Deposit''
records the number of iAsset in the SP deposited by all users.

A SP epoch ends when all iAsset from a SP is drained via liquidations.
Epoch is a running tally of the number of occurrences there have been
when the SP's total iAsset deposit reached zero. Upon updating the SP
state, if the total iAsset in the SP is to be set to zero, then this
marks the end of an epoch. At the end of an epoch, the following occurs:

\begin{itemize}
\item
  Epoch is recorded in a UTXO paired with the compounded constant value
  after the latest liquidation
\item
  The SP state is updated with the values:

  \begin{itemize}
  \item
    epoch incremented by one;
  \item
    product constant set to one; and
  \item
    compounded constant set to zero.
  \end{itemize}
\end{itemize}

\hypertarget{stability-pool-state}{}

\begin{tabularx}{\linewidth}{l|l}
\caption{State stored upon updates to a SP}
\tabularnewline
\toprule
\textbf{Name} & \textbf{Description}
\tabularnewline
\midrule
\endhead
\textbf{Product Constant} & The new product constant
(\(c\))
\tabularnewline
\midrule
\textbf{Compounded Constant} & The new compounded reward
(\(r\))
\tabularnewline
\midrule
\textbf{iAsset Deposit} & The updated amount of iAsset deposited into
the SP
\tabularnewline
\midrule
\textbf{Epoch} & The current epoch
\tabularnewline
\bottomrule
\end{tabularx}

\hypertarget{stability-deposits}{%
\begin{figure}[htbp]
\centering
\includesvg[width=8cm,inkscapelatex=false]{images/stability-deposits.svg}
\caption{iAsset being deposited into a new SP}
\label{stability-deposits}
\end{figure}}

When a user deposits iAsset into a SP, a SP staker ``account record'' is
created or updated for that user's account. The account record is
represented the same as SP state and stored within the UTXO of the SP
staker's position; iAsset Deposit records the number of iAssets owned
individually by the SP staker. All other values for the account record
are copied from the SP state.

\hypertarget{stability-liquidation}{%
\begin{figure}[htbp]
\centering
\includesvg[width=7cm,inkscapelatex=false]{images/stability-liquidation.svg}
\caption{SP state being updated after a liquidation occurs}
\label{stability-liquidation}
\end{figure}}

\filbreak

During a liquidation, iAsset is extracted from a SP. Proportionally, the
ownership share of the iAsset within each SP staker's position is
reduced. If the epoch in the account record matches the epoch in the SP
state, the amount of iAsset an individual SP staker holds can be
calculated using:

\[m = a\frac{c}{b}\]

Where:

\begin{itemize}
\item
  \(m\) is the amount of iAsset owed to the SP staker
\item
  \(a\) is the amount of iAsset the SP staker deposited (retrieved from
  the account record)
\item
  \(c\) is the current product constant (retrieved from the SP state)
\item
  \(b\) is the product constant when the SP staker deposited their
  iAsset (retrieved from the account record)
\end{itemize}

\hypertarget{stability-user-rewards}{%
\begin{figure}[htbp]
\centering
\includesvg[width=10cm,inkscapelatex=false]{images/stability-user-rewards.svg}
\caption{SP staker rewards after a liquidation has occurred}
\label{stability-user-rewards}
\end{figure}}

If the epoch in the account record does not match the epoch in the SP
state, then the amount of iAsset owned to the SP staker (\(m\)) is zero.
This is due to all the user's iAsset having been burned during a
previous epoch.

\hypertarget{stability-epoch}{%
\begin{figure}[htbp]
\centering
\includesvg[width=10cm,inkscapelatex=false]{images/stability-epoch.svg}
\caption{Illustration of a new SP epoch beginning after
a liquidation drains all iAsset}
\label{stability-epoch}
\end{figure}}

\hypertarget{stability-user-epoch-rewards}{%
\begin{figure}[htbp]
\centering
\includesvg[width=10cm,inkscapelatex=false]{images/stability-user-epoch-rewards.svg}
\caption{SP staker rewards after SP has been
drained and a new epoch has begun}
\label{stability-user-epoch-rewards}
\end{figure}}

During a liquidation, an ADA reward is deposited into the SP.
Proportionally, the share of ADA rewards each SP staker is owed
increases. The formula to calculate how much ADA an individual SP staker
is rewarded from the SP is:

\[l = a\frac{r - d}{b}\]

Where:

\begin{itemize}
\item
  \(l\) is the amount of ADA owed to the SP staker
\item
  \(a\) is the amount of iAsset the SP staker deposited (retrieved from
  the account record)
\item
  \(r\) is the current compounded constant (retrieved from the SP state
  or recorded compounded constant for the matching epoch)
\item
  \(d\) is the compounded constant when the SP staker deposited their
  iAsset (retrieved from the account record)
\item
  \(b\) is the product constant when the SP staker deposited their
  iAsset (retrieved from the account record)
\end{itemize}

If an account record's epoch does not match the epoch of the SP state,
then \(r\) is set to the latest recorded compounded constant for the
epoch. This is due to the compounding constant resetting to zero after
an epoch ends, therefore all SP staker positions during that epoch would
be closed because all their iAsset would have been utilized during
liquidations.

When a SP staker modifies their position, either by depositing or
withdrawing iAsset or ADA reward, then their previous position is
considered closed, and a new position is created. If a user withdraws
all their iAsset, then a new position is not opened. The SP state is
also updated to reflect the new deposit or withdrawal, i.e., the iAsset
Deposit is updated by the amount of iAsset deposited or withdrawn.

\hypertarget{stability-pool-staking-rewards}{%
\subsubsection{Stability Pool Staking
Rewards}\label{stability-pool-staking-rewards}}

SP stakers contribute to maintaining the solvency of the protocol and
the iAsset pegs. In return for staking their iAsset, Indigo offers
rewards in the form of ADA from liquidated CDPs and INDY.

INDY is rewarded each Cardano epoch (every five days) and determined by
the market cap of the iAsset as well as how much iAsset is being staked
relative to other iAssets. The less iAsset that is staked in a SP
relative to the total number of iAsset minted, the higher the INDY
reward; the more iAsset that's staked, the less the INDY reward.

\hypertarget{calculating-sp-indy-rewards}{%
\paragraph{Calculating SP INDY
rewards}\label{calculating-sp-indy-rewards}}

is broken into two phases:

\begin{enumerate}
\item
  Calculation of how much INDY is rewarded per SP
\item
  Calculation of SP staker's share of the SP's reward
\end{enumerate}

\begin{tabularx}{\linewidth}{l|r}
\caption{Distribution schedule of INDY unlocked every epoch for
Stability rewards}
\tabularnewline
\toprule
\textbf{Beginning From} & \textbf{\# INDY per Epoch}
\tabularnewline
\midrule
\endhead
1-Dec-22 & 28,768
\tabularnewline
\midrule
1-Dec-23 & 33,562
\tabularnewline
\midrule
26-Sep-24 & 33,561
\tabularnewline
\midrule
30-Nov-24 & 38,356
\tabularnewline
\midrule
30-Nov-25 & 43,150
\tabularnewline
\midrule
30-Nov-26 & 47,945
\tabularnewline
\bottomrule
\end{tabularx}

\hypertarget{calculating-indy-rewarded-per-sp}{%
\paragraph{Calculating INDY rewarded per
SP}\label{calculating-indy-rewarded-per-sp}}

is based on three variables:

\begin{enumerate}
\item
  Standard deviation of the SP's iAsset's underlying asset (\(\sigma\))
\item
  Stability Pool Saturation (\(\varphi\))
\item
  Market cap of the SP's iAsset (\(\omega\))
\end{enumerate}

\hypertarget{standard-deviation-for-a-sp-sigma}{%
\paragraph{\texorpdfstring{Standard deviation for a SP
(\(\sigma\))}{Standard deviation for a SP (\textbackslash sigma)}}\label{standard-deviation-for-a-sp-sigma}}

can be calculated using the formula:

\[\sigma = \left\{ \begin{matrix}
  \sqrt{\genfrac{}{}{}{}{\raisebox{5pt}{$\sum_{i = 1}^{\left| x \right|}\left( x_{i}\  - \ \overline{x} \right)^{2}$}}{\raisebox{-5pt}{$\left| x \right|$}}} & \text{if\ }\left| x \right| > 30 \\ \\
  0 & \text{if\ }\left| x \right| \leq 30 \\
  \end{matrix} \right.\ \]

Where:

\begin{itemize}
\item
  \(\sigma\) is the SP's standard deviation
\item
  \(x\) is the set of the iAsset's tracked asset's historical daily
  close price for the past year (or maximum amount of time that data
  exists for)
\end{itemize}

\filbreak

\hypertarget{stability-pool-saturation-varphi}{%
\paragraph{\texorpdfstring{Stability Pool Saturation
(\(\varphi\))}{Stability Pool Saturation (\textbackslash varphi)}}\label{stability-pool-saturation-varphi}}

can be calculated by taking the SP's deposits and dividing by the
iAsset's total supply:

\[\varphi = \left\{ \begin{matrix}
  \genfrac{}{}{}{}{\raisebox{5pt}{$\sum_{i = 1}^{\left| a \right|}a_{i}$}}{\raisebox{-5pt}{$b$}} & \text{if\ }c \geq d + 6 \\ \\
  0 & \text{if\ }c < d + 6 \\
  \end{matrix} \right.\ \]

Where:

\begin{itemize}
\item
  \(\varphi\) is the SP's saturation
\item
  \(a\) is the collection of iAsset deposits in the SP
\item
  \(b\) is the total supply of the iAsset
\item
  \(c\) is the current epoch number
\item
  \(d\) is the epoch number the iAsset was launched during
\end{itemize}

\filbreak

\hypertarget{market-cap-omega}{%
\paragraph{\texorpdfstring{Market cap
(\(\omega\))}{Market cap (\textbackslash omega)}}\label{market-cap-omega}}

can be calculated by taking the total number of the SP's iAsset minted
and multiplying by the price of the iAsset's tracked asset's price:

\[\omega = \left\{ \begin{matrix}
  ab & \text{if\ }c \geq d + 6 \\ \\
  0 & \text{if\ }c < d + 6 \\
  \end{matrix} \right.\ \]

Where:

\begin{itemize}
\item
  \(\omega\) is the SP's iAsset market cap
\item
  \(a\) is the total number of iAsset that have been minted
\item
  \(b\) is the price of the iAsset's tracked asset
\item
  \(c\) is the current epoch number
\item
  \(d\) is the epoch number the iAsset was launched during
\end{itemize}

\filbreak

\hypertarget{the-amount-of-indy-to-be-distributed-to-a-sp}{%
\paragraph{The amount of INDY to be distributed to a
SP}\label{the-amount-of-indy-to-be-distributed-to-a-sp}}

is calculated based on a value of \(\rho\), representing an average of
\(\sigma\), \(\varphi\), and \(\omega\), and is calculated daily using:

\[\rho = \left\{ \begin{matrix}
  \genfrac{}{}{}{}{\raisebox{5pt}{$\genfrac{}{}{}{}{\raisebox{7pt}{$1 / \sigma$}}{\raisebox{-7pt}{$\sum_{i = 1}^{\left| a \right|}\left\{ \begin{matrix}
    \\
    1 / a_{i} & \text{if\ }a_{i} > 0 \\ \\
    0 & \text{if\ }a_{i} = 0 \\ \\
    \end{matrix} \right.\ $}} + \genfrac{}{}{}{}{\raisebox{7pt}{$1 / \varphi$}}{\raisebox{-7pt}{$\sum_{i = 1}^{\left| b \right|}\left\{ \begin{matrix}
    \\
    1 / b_{i} & \text{if\ }b_{i} > 0 \\ \\
    0 & \text{if\ }b_{i} = 0 \\ \\
    \end{matrix} \right.\ $}} + \genfrac{}{}{}{}{\raisebox{7pt}{$\omega$}}{\raisebox{-7pt}{$\sum_{i = 1}^{\left| c \right|}c_{i}$}}$}}{\raisebox{-5pt}{$3$}} & if\ \sigma > 0\text{\ and\ }\varphi > 0\text{\ and\ }\omega > 0 \\
  & \\
  \genfrac{}{}{}{}{\raisebox{7pt}{$1 / \sigma$}}{\raisebox{-7pt}{$\sum_{i = 1}^{\left| a \right|}\left\{ \begin{matrix}
    \\
    1 / a_{i} & \text{if\ }a_{i} > 0 \\ \\
    0 & \text{if\ }a_{i} = 0 \\ \\
    \end{matrix} \right.\ $}} & \text{if\ }\sigma > 0\text{\ and\ }\varphi = 0\ and\ \omega = 0 \\
  & \\
  0 & \text{otherwise} \\
  \end{matrix} \right.\ \]

Where:

\begin{itemize}
\item
  \(\rho\) is the distribution value for the SP
\item
  \(a\) is the collection of each SP's \(\sigma\)
\item
  \(b\) is the collection of each SP's \(\varphi\)
\item
  \(c\) is the collection of each SP's \(\omega\)
\item
  \(\sigma\) is the SP's standard deviation
\item
  \(\varphi\) is the SP's saturation
\item
  \(\omega\) is the SP's iAsset market cap
\end{itemize}

\hypertarget{the-amount-of-indy-to-distribute-to-each-sp}{%
\paragraph{The amount of INDY to distribute to each
SP}\label{the-amount-of-indy-to-distribute-to-each-sp}}

can be calculated based off each SP's daily calculated \(\rho\):

\[a = \rho\frac{c}{5}\]

Where:

\begin{itemize}
\item
  \(a\) is the amount of INDY to distribute to the SP for a particular
  day within the epoch
\item
  \(\rho\) is the distribution value for the SP for a particular day
  within the epoch
\item
  \(c\) is the amount of INDY being distributed to all SPs for the epoch
\end{itemize}

\filbreak

\hypertarget{calculating-indy-reward-per-sp-staker}{%
\paragraph{Calculating INDY reward per SP
staker}\label{calculating-indy-reward-per-sp-staker}}

is based on the \(a\) for the user's SP and the amount of time the user
was staked in the SP:

\[k = \left\{ \begin{matrix}
  a \cfrac{b}{\sum_{i = 1}^{\left| c \right|}\left\{ \begin{matrix}
    \\
    c_{i} & \text{if\ }d_{i} \geq 24 \\ \\
    0 & \text{if\ }d_{i} < 24 \\ \\
    \end{matrix} \right.\ } & \text{if\ }e \geq 24 \\ \\
  0 & \text{if\ }e < 24 \\
  \end{matrix} \right.\ \]

Where:

\begin{itemize}
\item
  \(k\) is the amount of INDY to distribute to the SP staker for a
  particular day during the epoch
\item
  \(a\) is the amount of INDY to distribute to the SP for a particular
  day within the epoch
\item
  \(b\) is the total amount of iAsset staked by the SP staker
\item
  \(c\) is the collection of iAsset amounts staked by all SP stakers
\item
  \(d\) is the collection of hours all SP stakers have been staking
  iAsset for
\item
  \(e\) is the total hours the LP staker has been staking their LP
  tokens for
\end{itemize}

\hypertarget{total-amount-of-indy-to-distribute-to-the-sp-staker}{%
\paragraph{Total amount of INDY to distribute to the SP
staker}\label{total-amount-of-indy-to-distribute-to-the-sp-staker}}

is calculated by summing all the \(k\) values calculated during each day
of the epoch:

\[a = \left\{ \begin{matrix}
  \sum_{i = 1}^{5}b_{i} & \text{if\ }c \geq 24 \\ \\
  0 & \text{if\ }c < 24 \\
  \end{matrix} \right.\ \]

Where:

\begin{itemize}
\item
  \(a\) is the amount of INDY to distribute to the SP staker for the
  epoch
\item
  \(b\) is the collection of \(k\) values calculated each day of the
  epoch
\item
  \(c\) is the total hours the SP stakers has been staking iAsset for
\end{itemize}

SP stakers can withdraw their accumulated INDY staking rewards (the sum
of \(a\) for each epoch they're owed rewards) via the Indigo Web App.
Unclaimed rewards are withdrawable for three months. Any rewards not
claimed within three months after being rewarded are redistributed to
Members via the \protect\hyperlink{protocol-profit-sharing}{Collector}.

\hypertarget{oracles}{%
\subsection{Oracles}\label{oracles}}

To determine the value of collateral held within CDPs and the intended
prices of iAssets, Indigo makes use of Oracles\footnote{\href{https://chain.link/education/blockchain-oracles}{Oracles}
  provide a way for decentralized blockchain applications to access
  existing data sources.} available on Cardano. An Oracle queries
external data sources for information and makes that information
available on-chain.

Indigo is designed to be Oracle agnostic, meaning that it can support
any Oracle that publishes data on the Cardano blockchain so long as the
data format conforms with the protocol's specifications defined in the
\protect\hyperlink{cdp}{CDP} section.

\hypertarget{liquidity-staking-rewards}{%
\subsection{Liquidity Staking Rewards}\label{liquidity-staking-rewards}}

A benefit of iAsset composability is that they can be provided as
liquidity to any Decentralized Exchange (a ``DEX''). Having iAssets
available on several DEXs is a key factor to promote Indigo's
integration into the broader ecosystem, allowing other users to obtain
and use iAssets without having to manage a CDP.

Users who provide liquidity to DEXs receive tokens proving they have
deposited iAssets (a ``LP token''). Indigo rewards users who provide
iAsset liquidity by allowing them to stake their LP tokens in the
protocol and receive INDY rewards.

Stakers of LP tokens can unstake their tokens at any time. Members can
vote on whitelisting a specific LP token to be eligible for staking
rewards. Only double-sided LP tokens representing one iAsset and one
non-iAsset token in equal proportions are allowable and eligible for
rewards (e.g., iBTC/ADA LP token).

\begin{tabularx}{\linewidth}{l|r}
\caption{Distribution schedule of INDY unlocked every epoch for
Liquidity rewards}
\tabularnewline
\toprule
\textbf{Beginning From} & \textbf{\# INDY per Epoch}
\tabularnewline
\midrule
\endhead
21-Dec-22 & 4,795
\tabularnewline
\midrule
21-Dec-23 & 9,590
\tabularnewline
\midrule
11-Oct-24 & 9,589
\tabularnewline
\midrule
20-Dec-24 & 14,383
\tabularnewline
\midrule
20-Dec-25 & 19,178
\tabularnewline
\midrule
20-Dec-26 & 23,972
\tabularnewline
\bottomrule
\end{tabularx}

INDY is rewarded to LP stakers each epoch and determined by the market
cap of the iAsset as well as how much representative iAsset is being
staked relative to other iAssets. The less representative iAsset that is
staked in whitelisted LP tokens relative to the total number of iAsset
minted, the higher the INDY reward; the more representative iAsset
that's staked the less the INDY reward.

\hypertarget{calculating-liquidity-rewards}{%
\paragraph{Calculating Liquidity
rewards}\label{calculating-liquidity-rewards}}

is broken into two phases:

\begin{enumerate}
\item
  Calculation of how many INDY is rewarded per iAsset
\item
  Calculation of LP staker's share of the iAsset reward
\end{enumerate}

\hypertarget{calculating-liquidity-saturation-varphi}{%
\paragraph{\texorpdfstring{Calculating liquidity saturation
(\(\varphi\))}{Calculating liquidity saturation (\textbackslash varphi)}}\label{calculating-liquidity-saturation-varphi}}

requires taking the representative iAsset staked divided by the iAsset
total supply:

\[\varphi = \frac{\sum a}{b}\]

Where:

\begin{itemize}
\item
  \(\varphi\) is the liquidity saturation
\item
  \(a\) is the set of total iAsset staked for each pool corresponding
  with the set of whitelisted LP tokens for the iAsset
\item
  \(b\) is the total supply of the iAsset
\end{itemize}

\hypertarget{calculating-indy-rewarded-per-iasset}{%
\paragraph{Calculating INDY rewarded per
iAsset}\label{calculating-indy-rewarded-per-iasset}}

for a day during an epoch is based on assessing the liquidity saturation
comparative other iAssets in addition to the iAsset market caps:

\[k = \left\{ \begin{matrix}
  \frac{a}{5\left|b\right|} & \text{if\ }\varphi \geq 0.2\text{\ and\ }\varphi \leq 0.3 \\ \\
  \begin{pmatrix}
    \text{let\ }l = \sum_{i=0}^{\left| b \right|}\left\{\begin{matrix}
    \frac{a}{5\left|b\right|} & \text{if\ }b\geq 0.2 \text{\ and\ }b \leq 0.3         \\ \\
    0           & \text{otherwise}                                      \\
    \end{matrix}\right. \\ \\
    \text{let\ }m = \dfrac{1 / \varphi}{\sum_{i = 1}^{\left| b \right|}\left\{ \begin{matrix}
    0           & \text{if\ }b_{i} \geq 0.2\text{\ and\ }b_{i} \leq 0.3 \\ \\
    1 / b_{i}   & \text{otherwise}                                      \\
    \end{matrix} \right.}\\ \\
    \text{let\ }o = \dfrac{cd}{\sum_{i = 1}^{\left| b \right|}\left\{ \begin{matrix}
    0           & \text{if\ }b_{i} \geq 0.2\text{\ and\ }b_{i} \leq 0.3 \\ \\
    x_{i}y_{i}  & \text{otherwise}                                      \\
    \end{matrix} \right.} \\ \\
    \left( \frac{a}{5} - l\right) \left( m + o \right)
  \end{pmatrix} & \text{otherwise} \\
  \end{matrix} \right.\]

Where:

\begin{itemize}
\item
  \(k\) is the amount of INDY to distribute to the iAsset's LP stakers
  for a particular day within the epoch
\item
  \(a\) is the amount of INDY being distributed to all LP stakers for
  the epoch
\item
  \(b\) is the collection of each iAsset's \(\varphi\)
\item
  \(c\) is the intended price of the iAsset
\item
  \(d\) is the total supply of the iAsset
\item
  \(x\) is the collection of intended iAsset prices of the corresponding
  collection \(b\)
\item
  \(y\) is the collection of total iAsset supplies of the corresponding
  collection \(b\)
\end{itemize}

\hypertarget{indy-to-distribute-to-an-individual-lp-staker}{%
\paragraph{INDY to distribute to an individual LP
staker}\label{indy-to-distribute-to-an-individual-lp-staker}}

is calculated based on the staker's share of total iAsset staked:

\[r=k\frac{xy/z}{b\sum_{i=1}^{\left| a \right|}a_{i}b_{i}/c_{i}}\]

Where:

\begin{itemize}
\item
  \(r\) is the amount of INDY rewarded to the LP staker for a particular
  day within the epoch
\item
  \(k\) is the amount of INDY to distribute to the iAsset's LP stakers
  for a particular day within the epoch
\item
  \(a\) is the collection of staked amounts of LP tokens
\item
  \(b\) is the collection of total iAsset staked for pools corresponding
  with the LP tokens in collection \(a\)
\item
  \(c\) is the collection of total supply of the corresponding LP tokens
  collection \(a\)
\item
  \(x\) is the LP staker's amount LP tokens staked
\item
  \(y\) is the total iAsset staked in the LP staker's associated pool
\item
  \(z\) is the total supply of the LP tokens for the LP staker's
  associated pool
\end{itemize}

\(r\) is calculated daily for each user, and the sum of all \(r\) values
for each day is the amount of INDY the user is rewarded for the epoch:

\[a = \sum_{i = 1}^{5}b_{i}\]

Where:

\begin{itemize}
\item
  \(a\) is the amount of INDY rewarded to the LP staker the epoch
\item
  \(b\) is the collection of \(r\) values calculated for each day within
  the epoch
\end{itemize}

LP stakers can withdraw their accumulated INDY staking rewards (the sum
of \(a\) for each epoch they're owed rewards) via the Indigo Web App.
Unclaimed rewards are withdrawable for three months. Any rewards not
claimed within three months after being rewarded are redistributed to
Members via the \protect\hyperlink{protocol-profit-sharing}{Collector}.

\hypertarget{iasset-price-stability}{%
\subsection{iAsset Price Stability}\label{iasset-price-stability}}

\protect\hyperlink{synthetic-assets}{iAssets are pegged to tracked
assets}. To maintain price pegs, Indigo relies on protocol rules to
incentivize arbitrageurs and market forces to stabilize prices. These
rules ensure that iAssets are always fully collateralized, giving
further confidence to users that iAsset prices will match their
counterparts.

Periodically, Indigo receives price data from the outside world via
\protect\hyperlink{oracles}{Oracles}. The rate at which price feeds are
updated is configurable, and at launch will be set to once per hour.
After price is updated, CRs are adjusted across the protocol, allowing
for liquidations to occur for CDPs whose CR falls below the iAsset's
MCR.

If an iAsset drops in price relative to its peg, it provides CDP owners
an opportunity to buy the iAsset to repay their loan at a discount. This
can cause buying pressure on the iAsset to rise its price. If there is
an abundance of iAsset supply, Indigo can increase MCR towards the
iAsset mode CR.

Each CDP has its own CR. The iAsset mode CR represents the most frequent
CR value users select for their CDPs. By moving MCR towards the mode CR,
probability of liquidation increases, incentivizing users to close their
CDPs, which can cause iAsset buying pressure and reduced iAsset supply.

A higher MCR results in a higher cost to mint iAsset supply, reduces the
maximum leverage utilizable, and increases the margin of arbitrage value
for Stability Pool stakers. This creates a disincentive to create new
iAsset supply and incentivizes users to buy existing iAsset supply to
stake into the Stability Pool. The reduction of supply paired with the
increased buying pressure can push the iAsset price upwards.

When an iAsset is first launched, its supply is zero, yet its demand may
be high because users desire to purchase it. This causes an immediate
supply and demand imbalance, potentially causing the iAsset to trade
higher than its intended peg.

A low MCR reduces the cost of minting iAsset supply and maximizes the
leverage utilizable. This creates an incentive for users to create new
iAsset supply, hence is why Indigo's iAssets will initially be launched
with a MCR of 110\%. An iAsset MCR of 110\% forces the price of the
iAsset to be no more than 10\% above its peg by creating an arbitrage
opportunity. Users at any time can mint iAsset at a cost of 10\% higher
than the iAsset's pegged price, allowing iAsset to be immediately sold
if the market premium is higher than 10\%. iAsset trading above its peg
also offers an opportunity to borrow at a lower cost, further
incentivizing more supply to be minted and possibly creating additional
sell pressure if users choose to take advantage of the leverage.

If there is an abundance of iAsset demand and limited supply, Indigo can
decrease MCR towards 100\%. This in turn reduces the cost of minting
iAsset, pushing the price of the iAsset down. Indigo's quick liquidation
mechanism via \protect\hyperlink{stability-pools}{Stability Pools}
allows for high capital efficiency and support for very low
collateralization while still providing incentive for users to
participate in arbitrage. MCR value setting considers the average CR of
iAssets, ensuring that iAssets are always overcollateralized
irrespective of any market conditions or possible future events.

\hypertarget{governance}{%
\subsection{Governance}\label{governance}}

Governance is the decentralized voting process through which
\emph{proposals} for updating the protocol are introduced and either
accepted or rejected by the community (collectively known as the
``Indigo DAO''). All change to the protocol must go through governance.

Indigo has a 3-pillar structure built for long term sustainability:

\begin{enumerate}
\item
  \textbf{Indigo DAO --} Decentralized association of members governing
  the protocol.
\item
  \textbf{Indigo Foundation --} Foundation Company incorporated in the
  Cayman Islands for interacting with the real-world on behalf of the
  Indigo DAO.
\item
  \textbf{Indigo Laboratories, Inc. --} A Wyoming corporation contracted
  by the Indigo Foundation responsible for development of Indigo and
  blockchain technologies.
\end{enumerate}

\hypertarget{indigo-dao}{%
\subsubsection{Indigo DAO}\label{indigo-dao}}

The Indigo DAO (the ``DAO'') is an informal non-jurisdictional,
non-hierarchical, and nonprofit association of fluctuating individuals
and entities who are uncoordinated and act together using a token. The
DAO owns and controls the Indigo Protocol. All changes to the protocol
must go through governance. Governance is the decentralized voting
process through which proposals for updating the protocol are introduced
and either accepted or rejected by the Indigo DAO Members.

\protect\hyperlink{indy}{INDY} serves as Indigo DAO's utility token with
one of its purposes being to allow voting on DAO proposals. Users who
stake their INDY in Indigo's governance thereby become a DAO Member (a
``Member'') and can vote on proposals.

Members who wish to assist in managing the administrative and technical
operations of Indigo (e.g.: organizing meetings of Members, submitting
governance Proposals, or leading Working Groups) can be elected by other
Members and become Core Contributors.

\hypertarget{indigo-foundation}{%
\subsubsection{Indigo Foundation}\label{indigo-foundation}}

The Indigo Foundation (the ``Foundation'') entity provides an extremely
flexible framework that supports off-chain functions necessary for
executing the intent of the Indigo DAO. While the Indigo DAO is not a
legal entity, the Foundation is, and therefore can enter into legal
agreements with other real-world entities. The Foundation is established
to help implement approved actions of the DAO that cannot otherwise be
implemented in an automated or computational manner. The Foundation can
engage with governmental authorities (for tax, regulatory, or other
purposes), contract with vendors, and educate the community about Indigo
-- all as directed by the DAO.

The Foundation's authority is limited to implementing the votes of the
DAO and otherwise supporting Indigo. The DAO may vote to amend the
responsibilities of the Foundation at any time. The Foundation does not
have possession of or control of any Indigo or user funds. The DAO is
required to fund the Foundation and provide the Foundation with any
tokens needed to make payments to third party vendors.

\hypertarget{governance-process}{%
\subsubsection{Governance Process}\label{governance-process}}

An owner of INDY who chooses to stake INDY within Indigo becomes a
Member and obtains the right to vote on proposals. A vote can be either
in the form of yes, indicating favor of passing the proposal, or no,
indicating favor of rejecting the proposal. Each Member receives voting
power weighted by their amount of INDY staked.

The Governance Process consists of three phases.

\textbf{Step 1 -- Temperature Check}: A user creates and submits their
idea to the \href{https://forum.indigoprotocol.io/}{Indigo Forum}. The
idea will be reviewed by Moderators and Indigo Forum users for
consistency with the Indigo DAO Constitution. Forum users will review
and provide comments or suggested improvements to the idea, and
eventually vote on it within the Forum.

\textbf{Step 2 -- Proposal}: If a Temperature Check results in a
positive outcome, a user needs to deposit INDY to submit a proposal
on-chain. In addition, the user submitting the proposal should also
create \protect\hyperlink{governance-sharding}{voting shards} by
depositing some ADA. Voting shards will maintain a record of votes and
are meant to enhance on-chain voting performance. Members can vote on
the proposal using their staked INDY. Indigo's
\protect\hyperlink{adaptive-quorum-biasing}{Adaptive Quorum Biasing}
mechanism automatically adjusts the threshold to determine how many
positive votes are required for the proposal to pass.

\textbf{Step 3 -- Execution}: After a proposal's Voting Period ends, it
moves to the execution phase. If the proposal passed, users could
execute it and the proposal creator can retrieve their INDY deposit as
well as their ADA deposit within each voting shard.

If the proposal fails, the proposal is closed and the proposal creator
loses their INDY deposit. The INDY is instead sent to the
\protect\hyperlink{indigo-dao-treasury}{Treasury}.

\hypertarget{staking}{%
\subsubsection{Staking}\label{staking}}

Users who stake their \protect\hyperlink{indy}{INDY} in Indigo's
governance (thereby becoming a ``Member'') can vote on proposals. A vote
can be either in the form of \emph{yes}, indicating favor of passing the
proposal, or \emph{no}, indicating favor of rejecting the proposal. Each
INDY staker receives voting power weighted by their amount of INDY
staked and must either use either all or none of their voting power.

When a Member votes on a proposal, their INDY stake is locked until that
proposal's Voting Period has concluded (i.e., either approved, rejected,
or expired). Locked INDY cannot be withdrawn from the protocol. If a
Member votes on multiple proposals, their INDY is unlocked after the
most recently created proposal they voted on concludes. If their INDY is
in an unlocked state, then users can withdraw their INDY stake.

After casting a vote, it cannot be changed or undone. Voting power is
set to the total amount of INDY staked at the time of casting. If an
Member deposits additional INDY into their position, they can use that
INDY in addition to the existing locked INDY to vote on another proposal
but cannot use that additional INDY to vote on a proposal they've
already voted on. If the user deposits INDY after casting a vote and
before casting another vote, then the INDY can be withdrawn. After
depositing INDY and casting a vote for another proposal, all deposited
INDY becomes locked and cannot be withdrawn until the end of the
proposal.

\hypertarget{governance-rewards}{%
\subsubsection{Governance Rewards}\label{governance-rewards}}

Members who participate in Governance by casting a vote at least once
every ninety days (configurable itself by Member vote) are rewarded with
INDY each epoch. Each epoch, INDY is unlocked and distributed to all
qualifying Members. The amount of INDY each Member receives is based on
the ratio of a Member's stake relative to the total amount of INDY
staked, and can be calculated using:

\[a = \frac{bc}{\sum_{i = 1}^{\left| m \right|}m_{i}}\]

Where:

\begin{itemize}
\item
  \(a\) is the amount of INDY a Member is rewarded
\item
  \(b\) is the amount of INDY a Member has staked
\item
  \(c\) is the amount of INDY rewarded to all Members for the epoch
\item
  \(m\) is the collection of INDY amounts staked by all Members
\end{itemize}

\begin{tabularx}{\linewidth}{l|r}
\caption{Distribution schedule of INDY unlocked every epoch for
Governance rewards}
\tabularnewline
\toprule
\textbf{Beginning From} & \textbf{\# INDY per Epoch}
\tabularnewline
\midrule
\endhead
6-Dec-22 & 2,398
\tabularnewline
\midrule
6-Dec-23 & 3,596
\tabularnewline
\midrule
5-Dec-24 & 4,795
\tabularnewline
\midrule
13-Jul-25 & 4,794
\tabularnewline
\midrule
5-Dec-25 & 5,993
\tabularnewline
\midrule
5-Dec-26 & 7,191
\tabularnewline
\bottomrule
\end{tabularx}

\hypertarget{adaptive-quorum-biasing}{%
\subsubsection{Adaptive Quorum Biasing}\label{adaptive-quorum-biasing}}

A proposal is considered passed when the ratio of \emph{yes} votes over
\emph{no} votes exceeds the quorum threshold. Indigo uses a dynamic
vote-threshold mechanism called Adaptive Quorum Biasing (``AQB'') to
calculate the quorum threshold value. AQB lowers the quorum threshold as
more INDY is used to vote. If voter participation is low, then a high
majority of those votes must be in favor of the proposal. If voter
participation is high, then a lower majority of those votes must be in
favor of the proposal. Always at least 50\% of votes must be in favor of
a proposal for it to pass.

For example, if 29\% of all circulating supply of INDY is used to vote
during a proposal's Voting Period, the quorum threshold for that
proposal would be set to 66\%. This means that 66\% or more of the total
INDY used for voting would be required to vote \emph{yes} for the
proposal to be considered passed. If more than 34\% of the total INDY
used for voting voted \emph{no}, then the proposal would fail.

\hypertarget{quorum-threshold}{%
\begin{figure}[htbp]
\centering
\includesvg[width=17cm,inkscapelatex=false]{images/aqb.svg}
\caption{Illustration of quorum threshold decreasing as voter
participation increases}
\end{figure}}

\filbreak

To determine if a proposal is approved, the electorate (\emph{e}) first
needs to be calculated. \emph{e} is INDY circulating supply at the time
of a proposal's conclusion. INDY has a fixed distribution schedule, so
\emph{e} can be derived by taking the launch time of the protocol, the
end time of the proposal, and other values related to Indigo's token
distribution schedule set at the time of protocol launch.

To calculate \emph{e} the following logic can be used:

\[e = \begin{pmatrix}
  f:\left( a,b \right) \mapsto \begin{pmatrix}
  \text{let\ }x\text{\ equal\ }\min\left\{ \left\lfloor \frac{d - l}{5} \right\rfloor - a + 1,73\left| b \right| \right\} \\
  \\
  \left\{ \begin{matrix}
  \sum_{i = 1}^{x}\begin{pmatrix}
  \text{let\ }y\text{\ equal}\left\lfloor t\frac{b_{\left\lceil \frac{i}{73} \right\rceil}}{73} \right\rfloor \\
  \\
  \left\{ \begin{matrix}
  y + 1 & \text{if\ }i - 1 < \left\lfloor t\sum_{}^{}b - \sum_{j = 1}^{\left| b \right|}{73\left\lfloor \frac{tb_{j}}{73} \right\rfloor} \right\rfloor \\
  & \\
  y & \text{otherwise} \\
  \end{matrix} \right.\  \\
  \end{pmatrix} & \text{if\ }x > 0 \\
  & \\
  0 & \text{if\ }x \leq 0 \\
  \end{matrix} \right.\  \\
  \end{pmatrix} \\
  \\
  \text{let\ }z\text{\ equal}\begin{pmatrix}
  \text{let\ }x\text{\ equal\ }\min\left\{ \left\lfloor \frac{d - l - o}{365 \div 12} \right\rfloor + 1,q \right\} \\
  \\
  \left\{ \begin{matrix}
  0 & \text{if\ }x < 0 \\
  & \\
  \left\lfloor \frac{tp}{q} \right\rfloor & \text{if\ }x = 0\text{\ and\ }d - l \geq 0 \\
  & \\
  \left\lfloor \frac{xtp}{q} \right\rfloor & \text{otherwise} \\
  \end{matrix} \right.\  \\
  \end{pmatrix} \\
  \\
  \sum_{i = 1}^{\left| a \right|}{f\left( a_{i},b_{i} \right)} + z + \left\{ \begin{matrix}
  c & \text{if\ }d \geq l \\
  0 & \text{if\ }d < l \\
  \end{matrix} \right.\  \\
  \end{pmatrix}\]\\
Where:

\begin{itemize}
\item
  \(a\) is a set of delays for token distribution schedules (set at
  protocol launch)
\item
  \(b\) is a set of vesting distribution schedules (set at protocol
  launch)
\item
  \(c\) is the amount of INDY unlocked upon Indigo mainnet launch (set
  at protocol launch)
\item
  \(d\) is the date of the proposal's conclusion
\item
  \(l\) is the date of the first epoch after the launch of Indigo
  mainnet (set at protocol launch)
\item
  \(o\) is the offset for the start of Indigo's team distribution (set
  at protocol launch)
\item
  \(p\) is the percentage of INDY total supply allocated to the Indigo
  team (set at protocol launch)
\item
  \(q\) is the total number of months the Indigo team distribution lasts
  for (set at protocol launch)
\item
  \(t\) is the total supply of INDY (set at protocol launch)
\end{itemize}

\filbreak

Vesting schedules defined by \(b\) are represented as a set of sets
containing the percentage of token supply to be distributed per year,
with each value in the subset representing an individual year. For
example, consider the following set:

\[b = \left\{ \left\{ 0.01,0.02,0.03 \right\},\left\{ 0.05,0.1 \right\} \right\}\]

This defines two vesting schedules (two being the size of the set
\(b\)). The first vesting schedule in \(b\), referenced as \(b_{1}\),
describes a three-year vesting schedule (three being the size of the
subset \(b_{1}\)), with the first year distributing 1\% (0.01 being 1\%)
of total token supply, the second 2\%, and the third year 3\%, for a
total of 6\% (0.06 being the sum of all values in the subset \(b_{1}\))
of tokens distributed over the three years.

Knowing \(e\), a proposal's approval status can be calculated using the
formula:

\[q = \left\lfloor \frac{v_{y}}{\sqrt{e}} - \frac{v_{n}}{\sqrt{v_{y} + v_{n}}} \right\rfloor\]

Where:

\begin{itemize}
\item
  \(q\) is the vote threshold
\item
  \(e\) is the amount of INDY in circulation at time of the proposal's
  conclusion
\item
  \(v_{y}\) is the number of \emph{yes} votes
\item
  \(v_{n}\) is the number of \emph{no} votes.
\end{itemize}

If \(q\) is larger than 0, the proposal is passed. If \(q\) is equal to
or less than 0, the proposal is failed.

\hypertarget{governance-sharding}{%
\subsubsection{Governance Sharding}\label{governance-sharding}}

Upon creation of a proposal, multiple voting UTXOs can be created to
maintain records of votes. Each voting UTXO represents a shard. The
total number of shards that can be created is defined by the \emph{Total
Shards} protocol parameter.

After creating a proposal, the proposal's creator can create shards, up
to the number of Total Shards, by depositing ADA and submitting
transactions. If, after the proposal creation time plus the time defined
by the \emph{Proposing Period} protocol parameter, there are fewer
shards created than Total Shards, then the proposal is considered
expired.

The amount of ADA required to deposit to create an individual shard is
\(x\), as calculated and described in the
\protect\hyperlink{minimum-ada-to-create-utxo}{Minimum ADA to Create
UTXO section}. The proposal creator is required to deposit \(x\) ADA to
create an individual shard. To prevent a proposal from expiring before
all votes can be submitted, the proposal creator must deposit ADA
totaling \(x\) multiplied by Total Shards. The deposited ADA is later
returnable upon following correct voting procedures, as described in the
\protect\hyperlink{governance-proposal-process}{Governance Proposal
Process} section.

To vote, an INDY staker selects a shard to track their allocation. Each
shard records the total number of \emph{yes} and \emph{no} votes from
users who voted using that shard. A shard can only record a vote from
one user at a time. If a shard is in use by another user, then the user
must select an alternative shard to use. If all shards are in use, then
the user must wait until a shard becomes available.

At the end of the Voting Period, the shards can be closed. Upon closing,
all votes from each shard can be tallied, and the final vote counts can
be used to calculate whether the proposal has passed.

\filbreak

\hypertarget{governance-shards}{%
\begin{figure}[htbp]
\centering
\includesvg[width=10cm,inkscapelatex=false]{images/governance-shards.svg}
\caption{A voter selecting and casting their vote using a shard}
\label{governance-shards}
\end{figure}}

\hypertarget{governance-merge-shards-tally}{%
\begin{figure}[htbp]
\centering
\includesvg[width=10cm,inkscapelatex=false]{images/governance-merge-shards.svg}
\caption{Shards being merged to tally votes after a Voting Period
has ended}
\label{governance-merge-shards-tally}
\end{figure}}

\hypertarget{governance-proposal-types}{%
\subsubsection{Governance Proposal
Types}\label{governance-proposal-types}}

Users can submit the following type of proposals:

\begin{itemize}
\item
  \textbf{Whitelist an iAsset --} Propose that a new iAsset type be
  supported by the protocol. Attributes such as the MCR and Oracle price
  feed must be provided.
\item
  \textbf{Update an iAsset --} Propose that an existing iAsset's MCR
  and/or Oracle price feed be updated. Nullifying an iAsset's Oracle
  price feed causes that iAsset to be no longer mintable; therefore, it
  is delisted from the protocol.
\item
  \textbf{Text --} Propose that the Indigo DAO should adopt a proposal
  described textually. This formally records the DAO's intent on the
  blockchain but is not executed computationally, i.e., the proposal's
  executable message is non-actionable. A hash is stored on-chain, with
  the hash able to represent a Content Identifier (CID)\footnote{A CID
    is a self-describing
    \href{https://github.com/multiformats/cid}{content-addressed
    identifier} containing 32 characters. A CID can be used to
    \href{https://filecoin.tools/}{lookup data} stored on decentralized
    networks such as \href{https://filecoin.io/}{Filecoin}.} that
  references data on an external storage network.
\item
  \textbf{Upgrade Protocol --} Propose that the protocol should be
  upgraded to a new version.
\item
  \textbf{Update Protocol Parameters --} Propose that parameters
  describing protocol behavior be updated. Updateable parameters are
  shown in the \protect\hyperlink{protocol-parameters}{Protocol
  Parameters} table.
\end{itemize}

\hypertarget{protocol-parameters}{%
\subsubsection{Protocol Parameters}\label{protocol-parameters}}

Protocol parameters are updateable via proposals and define some
behaviors of the protocol. They exist as a map of values inside a UTXO.
Users and protocol functions can reference the values of the latest
defined protocol parameters to utilize within transactions.

\begin{tabularx}{\linewidth}{l|L}
\caption{Parameters that are updateable via an Update Protocol
Parameters Governance Proposal}
\tabularnewline
\toprule
\textbf{Parameter Name} & \textbf{Description}
\tabularnewline
\midrule
\endhead
\textbf{Effective Delay} & The number of seconds after a passed proposal
closes before it becomes eligible for execution.
\tabularnewline
\midrule
\textbf{Expiration Period} & The maximum number of seconds allowed after
a passed proposal closes for it to be executed. If the proposal isn't
executed in time, then the proposal is considered
expired.
\tabularnewline
\midrule
\textbf{Proposal Deposit} & The amount of INDY that is required to be
deposited to create a proposal. If a proposal passes, the INDY deposit
is returnable to the owner. If a proposal fails, the INDY deposit is
non-returnable, and instead is only transferable to the
\protect\hyperlink{indigo-dao-treasury}{DAO Treasury}.
\tabularnewline
\midrule
\textbf{Proposing Period} & The maximum number of seconds allowed after
a proposal is created for its shards to be created. If shards are not
created by this time then the proposal fails and the creator loses their
deposit.
\tabularnewline
\midrule
\textbf{Protocol Fee Percentage} & The percentage of ADA to take as a
protocol fee when withdrawing collateral from CDPs or redeeming SPL
rewards.
\tabularnewline
\midrule
\textbf{Total Shards} & The total number of
\protect\hyperlink{governance-sharding}{Governance Shards} to utilize
during Voting Periods of proposals.
\tabularnewline
\midrule
\textbf{Voting Period} & The number of seconds a proposal remains open
for voting after being created.
\tabularnewline
\bottomrule
\end{tabularx}

\hypertarget{governance-proposal-process}{%
\subsubsection{Governance Proposal
Process}\label{governance-proposal-process}}

Any user can create a proposal by depositing a fixed amount of INDY into
the protocol. The amount of INDY required is determined by the value of
the \emph{INDY Deposit} protocol parameter.

Once submitted, the proposal becomes eligible for the proposal creator
to create shards. After one or more shards are created for a proposal,
it can be voted on by INDY stakers until that proposal's Voting Period
has concluded.

Proposals are recorded on-chain with an executable message encoding the
specific effects of each one. Upon execution, the proposal will be
processed with the full privileges of the governance contracts.

The following steps outline the proposal lifecycle:

\begin{enumerate}
\item
  A user creates a new proposal by depositing an amount of INDY that
  equals the Proposal Deposit.
\item
  The proposal creator creates one or more shards, up to a maximum of
  Total Shards, by depositing ADA. All shards must be created before the
  Proposing Period ends for the proposal to pass.
\item
  The proposal enters the voting phase, where INDY stakers can vote
  (\emph{yes}/\emph{no}) using their staked INDY positions. INDY of the
  INDY stakers who vote remains locked until the \emph{Voting Period}
  ends.
\item
  The Voting Period ends after more time has passed than the proposal's
  creation time, plus time defined by the Voting Period protocol
  parameter.
\item
  After the Voting Period has ended, the proposal can be closed by its
  creator.
\item
  If the proposal passes, its executable contents can be executed by
  users after a delay defined by the \emph{Effective Delay} protocol
  parameter. The proposal must be executed prior to the time described
  by the \emph{Expiration Period} protocol parameter; otherwise, the
  proposal will be considered expired and no longer executable.
\end{enumerate}

Several actions can be taken against a proposal by users:

\begin{itemize}
\item
  \textbf{Create --} Creates a proposal conforming to one of the allowed
  \protect\hyperlink{governance-proposal-types}{Governance Proposal
  Types}.
\item
  \textbf{Create Shard --} The owner of the proposal is expected to --
  and can -- create one or more shards, up to a maximum of \emph{Total
  Shards}. For a proposal to be eligible to pass, the number of shards
  created must equal Total Shards. A shard is created by depositing ADA
  alongside a request to create one. Shards can only be created from the
  creation of the proposal up until the \emph{Proposing Period} ends.
  Creating shards after the Proposing Period will cause the transaction
  to fail.
\item
  \textbf{Merge Shards --} Users can merge two or more shards created
  after the proposal's Voting Period ends and before the proposal is
  closed. Upon merging, the owner is eligible to receive back the ADA
  that was deposited to create each merged shard after the proposal is
  closed.
\item
  \textbf{Close --} The owner of the proposal can close the proposal
  after its Voting Period ends if the number of shards created is equal
  to Total Shards, and after all shards have been merged. If the number
  of shards created is less than Total Shards, then the proposal cannot
  be closed until after the proposal expires. After the owner closes
  their proposal, they receive back any ADA that was deposited to create
  each shard. If a proposal expires before the owner closes the
  proposal, then any user can close the proposal.
\item
  \textbf{Execute --} If a proposal is closed and has passed, any user
  can execute it. Upon execution, the protocol runs the executable
  message embedded within the proposal to apply changes to the protocol.
\end{itemize}

A proposal has the following states:

\begin{itemize}
\item
  \textbf{Created --} After a proposal is created it is available for
  the owner to create shards.
\item
  \textbf{Open --} When a proposal has one or more shards available then
  it becomes available for INDY holders to vote on. If a proposal has at
  least one shard but less than T\emph{otal Shards}, the proposal is
  \emph{Open}.
\item
  \textbf{Active --} When a proposal has shards that equal \emph{Total
  Shards}, all shards were created before the Proposing Period, and time
  has not exceeded its \emph{Voting Period}, then the proposal is
  \emph{Active}.
\item
  \textbf{Ended --} When a proposal has exceeded its \emph{Voting
  Period}, then the proposal is \emph{Ended}.
\item
  \textbf{Merged --} When all the proposal's shards have been merged,
  then the proposal is \emph{Merged}.
\item
  \textbf{Closed --} When a proposal is \emph{Ended,} and after a user
  has made a submission for the proposal to close, then the proposal is
  \emph{Closed}.
\item
  \textbf{Passed --} When a proposal is \emph{Closed} and the number of
  \emph{yes} votes exceeds the quorum threshold, then the proposal is
  \emph{Passed}.
\item
  \textbf{Failed --} When a proposal is \emph{Closed} and the number of
  \emph{yes} votes does not exceed the quorum threshold, then the
  proposal is \emph{Failed}.
\item
  \textbf{Expired --} When a proposal has exceeded its \emph{Execution
  Period} without being executed, if the created shards are fewer than
  Total Shards after the Voting Period, or if shards have been created
  after the Proposing Period, then the proposal is \emph{Expired}.
\item
  \textbf{Executed --} When a proposal is \emph{Passed} and not
  \emph{Expired}, then any user can execute the proposal. The proposal
  then becomes \emph{Executed}.
\end{itemize}

\hypertarget{governance-proposal-lifecycle}{%
\begin{figure}[htbp]
\centering
\includesvg[width=18cm,inkscapelatex=false]{
images/proposal-process-diagram.svg}
\caption{Illustration of the proposal lifecycle}
\end{figure}}

\hypertarget{indigo-dao-treasury}{%
\subsubsection{Indigo DAO Treasury}\label{indigo-dao-treasury}}

The Indigo DAO owns and controls a DAO Treasury (the ``Treasury''). Upon
minting of INDY, a portion of INDY (the amount is defined at protocol
launch) is sent to the Treasury. The INDY in the Treasury is intended to
be used for future versions of the protocol and controlled by the
\protect\hyperlink{governance-1}{governance process}.

To permanently identify the Indigo DAO on the Cardano blockchain, a NFT
is minted as the official Indigo DAO identity token (``identity token'')
and held in the Treasury. The identity token is transferred to wherever
the latest version of the Treasury lives. The protocol transfers the
identity token and INDY in the Treasury upon future protocol upgrades.

\hypertarget{protocol-upgrade}{%
\subsubsection{Protocol Upgrade}\label{protocol-upgrade}}

Indigo is designed to be continually and incrementally upgraded. Instead
of releasing distinct protocols that users may interact with
individually, the Indigo Protocol exists as a singular protocol whose
underlying validators may periodically be updated. From a user's
perspective, the interaction is seamless, since they will only interact
with one protocol, regardless of the version of Indigo Protocol that is
live on the Cardano blockchain.

A single protocol has been launched, and new features will be added to
Indigo via approval from Members. Protocol upgrades are driven by the
\protect\hyperlink{governance}{governance process}. To suggest new
features, a Text proposal and development request must first be approved
and authorized by the DAO. A development firm such as Indigo
Laboratories will then begin work on building software to implement the
new features.

When software is ready for deployment, a request to upgrade is submitted
to Indigo. Members can inspect the new code requested to be deployed and
either approve or reject the proposal. Upon approval, the developing
entity of the software can deploy the code to Cardano, and Indigo will
be automatically upgraded to a new version. However, the code must match
the code approved by Members, otherwise Indigo will not recognize the
new features as authentic, and no upgrade will take place.

After deployment and approval, individual user positions can be migrated
from the old version to the new one. Some features may not be available
until the user has migrated their positions. For example, if a user owns
a CDP, they will be unable to add collateral to their CDP until they
migrate their CDP to the new version of Indigo. To migrate a CDP, a user
will have to pay a small transaction fee in the form of ADA and submit
the migration request via the Indigo Web App. If a user chooses not to
migrate a CDP, they will not be able to deposit more collateral or mint
more iAsset; their CDP may become at risk of liquidation. Another user
may opt to migrate a CDP subject to liquidation to perform the
liquidation and confiscate the underlying collateral, with the original
CDP owner losing their collateral.

\hypertarget{upgrade-protocol}{%
\begin{figure}[htbp]
\centering
\includesvg[width=17cm,inkscapelatex=false]{images/upgrade-protocol.svg}
\caption{Illustration of an Upgrade Protocol proposal upgrading the
CDP, SP, and Collector contracts}
\end{figure}}

\hypertarget{migrate-utxo}{%
\begin{figure}[htbp]
\centering
\includesvg[width=17cm,inkscapelatex=false]{images/migrate-utxo.svg}
\caption{Illustration of a UTXO migrating from an old validator to a
new validator}
\end{figure}}

\hypertarget{upgrade-protocol-flowchart}{%
\begin{figure}[htbp]
\centering
\includesvg[width=14cm,inkscapelatex=false]{images/upgrade-protocol-process-diagram.svg}
\caption{The process to upgrade the protocol}
\end{figure}}

\hypertarget{protocol-profit-sharing}{%
\subsection{Protocol Profit Sharing}\label{protocol-profit-sharing}}

As users create and close CDPs, and as CDPs are liquidated, a fee is
collected. Members are rewarded by receiving a share of the collected
fees. The fee is set to 2\% and modifiable by vote of Members.

When a fee is collected, it is sent to the Collector smart contract. The
Collector's purpose is to collect protocol fees and distribute them to
INDY stakers. Users who stake their INDY are eligible to a share of all
collected protocol fees, proportional to their share of total INDY
staked.

The Collector maintains a collection of UTXOs that can be used to store
ADA. When a protocol fee is collected, such as during withdrawal of a
liquidation reward, the user selects a UTXO from the Collector to send
the fee to. The amount of ADA required to deposit to create a Collector
UTXO is \(x\), as calculated and described in the
\protect\hyperlink{minimum-ada-to-create-utxo}{Minimum ADA to Create
UTXO section}. A Collector UTXO can be created by any user who deposits
\(x\) ADA. Once deposited, a new UTXO is added to the Collector and the
ADA cannot be withdrawn.

\hypertarget{collector-utxo}{%
\begin{figure}[htbp]
\centering
\includesvg[width=10cm,inkscapelatex=false]{images/collector.svg}
\caption{A user paying a fee to the Collector}
\label{collector-utxo}
\end{figure}}

Users can request to gather fees from Collector UTXOs and collectively
send them to the Staking Manager who is responsible for allowing INDY
stakers to withdraw their share of owed fees, and will only accept
deposits of fees if there are one or more INDY stakers. If no INDY is
staked, then user requests to transfer fees from the Collector to the
Staking Manager will fail.

\hypertarget{collector-staking-manager}{%
\begin{figure}[htbp]
\centering
\includesvg[width=10cm,inkscapelatex=false]{images/collector-staking-manager.svg}
\caption{Transferring collected fees to the Staking Manager}
\label{collector-staking-manager}
\end{figure}}

\filbreak

The Staking Manager keeps track of the number of INDY that are staked as
well as a snapshot value. The snapshot value is a running total (with a
precision of six decimals) of reward deposits updated each time ADA is
transferred from the Collector to the Staking Manager, and can be
calculated using:

\[a = b + \frac{c}{d}\]

Where:

\begin{itemize}
\item
  \(a\) is the new snapshot value to be stored by the Staking Manager,
  truncated to six decimals
\item
  \(b\) is the current snapshot value stored by the Staking Manager
\item
  \(c\) is the amount of ADA deposited into the Staking Manager from the
  Collector
\item
  \(d\) is the total amount of INDY staked in the Staking Manager
\end{itemize}

The snapshot value is initially set to zero. When a user stakes INDY,
the current snapshot value is stored in the INDY staker's position, and
the total amount of INDY staked is updated in the Staking Manager. When
an INDY staker updates or closes their position, all rewards are
withdrawn. INDY staker rewards can be calculated using:

\[a = d\left( b - c \right)\]

Where:

\begin{itemize}
\item
  \(a\) is the amount of ADA reward the user is owed
\item
  \(b\) is the current snapshot value stored by the Staking Manager
\item
  \(c\) is the snapshot value when the user staked their INDY
\item
  \(d\) is the amount of INDY the user has staked
\end{itemize}

\hypertarget{smart-contract-design}{%
\section{Smart Contract Design}\label{smart-contract-design}}

In Cardano's eUTXO model\footnote{\href{https://docs.cardano.org/learn/eutxo-explainer}{Cardano
  utilizes the eUTXO model} to perform arbitrary logic permitted by
  smart contracts.}, each transaction has inputs and outputs. An input
is a UTXO that is an output of another transaction. Users interact with
the protocol by performing actions and submitting transactions
containing those actions to Protocol Endpoints. Submitted transactions
are validated by the protocol's smart contracts (also known as
validators). If a transaction is successfully validated (i.e.,
permitted), then an action is put into effect by the transaction's
execution.

\hypertarget{diagram-legend}{%
\begin{figure}[htbp]
\centering
\includesvg[width=9cm]{images/endpoints/legend.svg}
\caption{Legend for Protocol Endpoint transaction examples}
\end{figure}}

Protocol Endpoints allow users to interact with the protocol by
performing a specific action such as
\protect\hyperlink{collateralized-debt-positions}{opening a CDP},
\protect\hyperlink{governance-proposal-process}{submitting a proposal},
\protect\hyperlink{stability-pools}{depositing iAssets in a SP}, etc. A
Protocol Endpoint can take input in the form of UTXOs. Input is provided
either by consuming or referencing. To consume a UTXO is to spend the
UTXO in whole within the transaction. By consuming the UTXO it allows
change to the state of that UTXO, such as updating the balance. To
reference a UTXO is to read the UTXO without change. Only one user can
consume a single UTXO at a time, whereas many users can simultaneously
reference UTXOs.

Protocol Endpoints may perform actions in the form of minting or
burning. Minting a token creates a new token and allows it to be used as
input. Upon minting, the token may be stored in a UTXO containing datum
that can be read for additional information. Burning a token destroys an
existing token, making it no longer usable as input.

Outputs are UTXOs that are created as an effect of a transaction. For
example, a Protocol Endpoint may create an output to represent a user
position or a pool of tokens. After an output is created it can be used
as an input.

Following are details for each Indigo smart contract, their tokens
issued, parameter inputs, and outputs.

For the described smart contract parameters, token types are in the form
of \(Value.AssetClass\)\footnote{\href{https://playground.plutus.iohkdev.io/doc/haddock/plutus-ledger-api/html/Plutus-V1-Ledger-Value.html\#g:3}{An
  asset class} is identified by currency symbol and token name.}. The
smart contracts look for the UTXO with the token type and may read the
datum of that UTXO for additional information.

\hypertarget{cdp}{%
\subsection{CDP}\label{cdp}}

The CDP contracts are used to store the collateral used to mint iAssets.
There are two contracts for managing CDPs: CDPCreator and CDP. The
CDPCreator validates the creation of a user's CDP UTXO. The CDP contract
is used to manage a user's individual position by validating actions
such as storing collateral, minting iAssets, and performing SPL.

\begin{tabularx}{\linewidth}{l|L|L}
\caption{CDP native tokens}
\tabularnewline
\toprule
\textbf{Name} & \textbf{Description} & \textbf{Minting
Policy}
\tabularnewline
\midrule
\endhead
\textbf{CDPCreatorNFT}
&
Identifies the authentic CDPCreator output

Validators ensure that this NFT always stays at the CDPCreator
output
&
The protocol mints more than 1 token at initialization
\tabularnewline
\midrule
\textbf{CDPToken} & Identifies an authentic CDP output & The transaction
must spend CDPCreatorNFT or consume a CDPToken
\tabularnewline
\midrule
\textbf{iAssetToken}
&
Identifies an authentic iAsset output, where datum is stored defining
iAsset information including the OracleAssetNFT used to reference the
latest price

Validators ensure that this token always stays at an iAsset output
&
The transaction must consume GovNFT
\tabularnewline
\midrule
\textbf{iAssets (iBTC, iETH, etc.)} & Synthetic version of BTC, ETH,
etc. & The transaction must consume a CDPToken
\tabularnewline
\bottomrule
\end{tabularx}

\begin{tabularx}{\linewidth}{l|L|L}
\caption{CDP token inputs}
\tabularnewline
\toprule
\textbf{Type} & \textbf{Description} & \textbf{Datum}
\tabularnewline
\midrule
\endhead
\textbf{OracleAssetNFT}
&
The NFT managed by an Oracle provider that's used to record price
information for an iAsset
&
\emph{odPrice}: The price with six decimals of precision

\emph{odExpiration}: The timestamp in which the oracle price
expires
\tabularnewline
\bottomrule
\end{tabularx}

\hypertarget{cdpcreator-parameters}{%
\subsubsection{CDPCreator Parameters}\label{cdpcreator-parameters}}

\begin{itemize}
\item
  \texttt{cdpCreatorNFT~::~CDPCreatorNFT}. NFT for identifying authentic
  CDPCreator output.
\item
  \texttt{cdpAssetCs~::~CurrencySymbol}. Currency symbol for the minting
  policy of iAssets.
\item
  \texttt{cdpAuthTk~::~CDPToken}. Token for identifying authentic CDP
  output.
\item
  \texttt{iAssetAuthTk~::~iAssetToken}. Token for identifying authentic
  iAsset output including datum with the iAsset name, MCR, and
  OracleAssetNFT reference to find the latest price for the asset.
\item
  \texttt{versionRecordToken~::~VersionRecordToken}. Token for
  identifying the version record for a protocol upgrade.
\item
  \texttt{cdpScriptHash~::~ValidatorHash}. Hash of CDP script, used for
  verifying the output of a CDP.
\end{itemize}

\hypertarget{cdp-parameters}{%
\subsubsection{CDP Parameters}\label{cdp-parameters}}

\begin{itemize}
\item
  \texttt{cdpAuthToken~::~CDPToken}. Token for identifying authentic CDP
  output.
\item
  \texttt{cdpAssetSymbol~::~CurrencySymbol}. Currency symbol for the
  minting policy of iAssets.
\item
  \texttt{iAssetAuthToken~::~iAssetToken}. Token for identifying
  authentic iAsset output.
\item
  \texttt{stabilityPoolAuthToken~::~StabilityPoolToken}. Token
  identifying authentic SP output.
\item
  \texttt{versionRecordToken~::~VersionRecordToken}. Token for
  identifying the version record for a protocol upgrade.
\item
  \texttt{upgradeToken~::~UpgradeToken}. Token for identifying proposal
  Upgrade tokens to update iAsset output.
\item
  \texttt{collectorValHash~::~ValidatorHash}. The validator hash for the
  Collector contract.
\item
  \texttt{govNFT~::~GovNFT}. NFT for identifying authentic governance
  parameters.
\item
  \texttt{spValHash~::~ValidatorHash}. The validator hash for the SP
  contract.
\end{itemize}

\begin{tabularx}{\linewidth}{l|L|L|L}
\caption{CDP outputs}
\tabularnewline
\toprule
\textbf{Type} & \textbf{Description} & \textbf{Datum} &
\textbf{Values}
\tabularnewline
\midrule
\endhead
\textbf{CDPCreator}
&
Many CDPCreator outputs exist for the protocol

To create a CDP output, this output must be consumed
&

&
\emph{CDPCreatorNFT}: 1
\tabularnewline
\midrule
\textbf{CDP}
&
Each CDP output represents an individual position
&
\emph{cdpOwner}: The public key hash that owns this CDP

\emph{cdpIAsset}: The type of iAsset associated with this CDP

\emph{cdpMintedAmount}: Amount of iAsset minted from this position
&
\emph{CDPToken}: 1

\emph{ADA}: collateral locked in this position
\tabularnewline
\midrule
\textbf{iAsset}
&
Each iAsset output represents an iAsset
&
\emph{iaName}: the name of iAsset

\emph{iaMinRatio}: The minimum collateral ratio of iAsset

\emph{iaPrice}: Either the final price for the delisted asset or the
OracleAssetNFT used to reference the price feed
&
\emph{iAssetToken}: 1
\tabularnewline
\bottomrule
\end{tabularx}

\hypertarget{cdp-endpoints}{%
\subsubsection{CDP Endpoints}\label{cdp-endpoints}}

\hypertarget{cdp-open}{%
\subparagraph{CDP: Open}\label{cdp-open}}

Creates a CDP associated with an iAsset type

\begin{tabularx}{\linewidth}{l|l|L}
\toprule
\textbf{Type} & \textbf{Amount} & \textbf{Description}
\tabularnewline
\midrule
\endhead
\textbf{Redeemer} & N.A. & CreateCDP, takes as parameters a public key
hash corresponding to a user's wallet, amount of iAssets to mint, and
ADA collateral to deposit
\tabularnewline
\midrule
\textbf{Consume} & 1 & CDPCreator UTXO
\tabularnewline
\midrule
\textbf{Consume} & 1+ & ADA to be used as collateral
\tabularnewline
\midrule
\textbf{Reference} & 1 & iAsset UTXO that identifies the iAsset to
mint
\tabularnewline
\midrule
\textbf{Reference} & 1 & UTXO containing the OracleAssetNFT with a datum
describing the iAsset price
\tabularnewline
\midrule
\textbf{Mint} & \(\infty\) & The minted iAsset tokens (dependent on the
ADA deposited, iAsset MCR determined from the iAsset UTXO, and iAsset
price)
\tabularnewline
\midrule
\textbf{Mint} & 1 & CDPToken that identifies a user's
position
\tabularnewline
\midrule
\textbf{Output} & 1 & CDPCreator UTXO
\tabularnewline
\midrule
\textbf{Output} & 1 & CDP UTXO that represents a user's
CDP
\tabularnewline
\midrule
\textbf{Output} & 1 & The UTXO sent to the user's wallet containing the
minted iAsset
\tabularnewline
\bottomrule
\end{tabularx}

\hypertarget{cdp-open-figure}{%
\begin{figure}[htbp]
\centering
\includesvg[width=17cm]{
images/endpoints/CDP/CDP-Open-Position.svg}
\caption{Example of creating a CDP with 500 ADA and minting 100 iAsset}
\label{cdp-open-figure}
\end{figure}}

\hypertarget{cdp-deposit-collateral}{%
\subparagraph{CDP: Deposit Collateral}\label{cdp-deposit-collateral}}

Deposit ADA collateral into an existing CDP

\begin{tabularx}{\linewidth}{l|l|L}
\toprule
\textbf{Type} & \textbf{Amount} & \textbf{Description}
\tabularnewline
\midrule
\endhead
\textbf{Redeemer} & N.A. & AdjustCDP
\tabularnewline
\midrule
\textbf{Consume} & 1 & CDP UTXO that represents the user's current
position
\tabularnewline
\midrule
\textbf{Consume} & 1+ & UTXOs containing ADA from the user's wallet to
be used as collateral
\tabularnewline
\midrule
\textbf{Reference} & 1 & iAsset UTXO that serves to identify the iAsset
that the CDP is for
\tabularnewline
\midrule
\textbf{Output} & 1 & CDP UTXO that represents the user's adjusted
CDP
\tabularnewline
\midrule
\textbf{Output} & 1 & New UTXO to the user wallet returning change (if
any)
\tabularnewline
\bottomrule
\end{tabularx}

\hypertarget{cdp-deposit-collateral-figure}{%
\begin{figure}[htbp]
\centering
\includesvg[width=17cm]{
images/endpoints/CDP/CDP-Deposit-Collateral.svg}
\caption{Example of depositing an additional 500 ADA into an
existing CDP}
\label{cdp-deposit-collateral-figure}
\end{figure}}

\hypertarget{cdp-withdraw-collateral}{%
\subparagraph{CDP: Withdraw Collateral}\label{cdp-withdraw-collateral}}

Withdraw ADA collateral from an existing CDP

\begin{tabularx}{\linewidth}{l|l|L}
\toprule
\textbf{Type} & \textbf{Amount} & \textbf{Description}
\tabularnewline
\midrule
\endhead
\textbf{Redeemer} & N.A. & AdjustCDP
\tabularnewline
\midrule
\textbf{Consume} & 1 & CDP UTXO that represents the user's current
position
\tabularnewline
\midrule
\textbf{Consume} & 1 & Collector UTXO that may already contain fees
previously collected
\tabularnewline
\midrule
\textbf{Reference} & 1 & iAsset UTXO that serves to identify the iAsset
should be minted
\tabularnewline
\midrule
\textbf{Reference} & 1 & UTXO containing the OracleAssetNFT with a datum
describing the iAsset price
\tabularnewline
\midrule
\textbf{Output} & 1 & CDP UTXO that represents the user's adjusted
position
\tabularnewline
\midrule
\textbf{Output} & 1 & Collector UTXO that contains a portion of the
withdrawn collateral (taken as a fee)
\tabularnewline
\midrule
\textbf{Output} & 1 & A new UTXO to the user wallet containing the
withdrawn collateral
\tabularnewline
\bottomrule
\end{tabularx}

\filbreak

\hypertarget{cdp-withdraw-collateral-figure}{%
\begin{figure}[htbp]
\centering
\includesvg[width=17cm]{
images/endpoints/CDP/CDP-Withdraw-Collateral.svg}
\caption{Example of withdrawing 500 ADA from a CDP and paying a 10
ADA fee}
\label{cdp-withdraw-collateral-figure}
\end{figure}}

\hypertarget{cdp-close}{%
\subparagraph{CDP: Close}\label{cdp-close}}

Closes an existing CDP

\begin{tabularx}{\linewidth}{l|l|L}
\toprule
\textbf{Type} & \textbf{Amount} & \textbf{Description}
\tabularnewline
\midrule
\endhead
\textbf{Redeemer} & N.A. & CloseCDP
\tabularnewline
\midrule
\textbf{Redeemer} & N.A. & Collect
\tabularnewline
\midrule
\textbf{Consume} & 1 & CDP UTXO that represents the user's current
position
\tabularnewline
\midrule
\textbf{Consume} & 1 & Collector UTXO that may already contain fees
previously collected
\tabularnewline
\midrule
\textbf{Consume} & 1+ & UTXOs from the user's wallet containing iAsset
tokens of the same type as the CDP
\tabularnewline
\midrule
\textbf{Reference} & 1 & iAsset UTXO that serves to identify the iAsset
the CDP is for
\tabularnewline
\midrule
\textbf{Reference} & 1 & UTXO containing the OracleAssetNFT with a datum
describing the iAsset price
\tabularnewline
\midrule
\textbf{Burn} & \(\infty\) & iAssets that were sent by the
user
\tabularnewline
\midrule
\textbf{Burn} & 1 & CDPtoken
\tabularnewline
\midrule
\textbf{Output} & 1 & Collector UTXO that contains a portion of the CDP
collateral (taken as a fee)
\tabularnewline
\midrule
\textbf{Output} & 1 & A new UTXO to the user wallet containing the total
collateral (minus the fee)
\tabularnewline
\bottomrule
\end{tabularx}

\hypertarget{cdp-close-figure}{%
\begin{figure}[htbp]
\centering
\includesvg[width=17cm]{
images/endpoints/CDP/CDP-Close-Position.svg}
\caption{Example of closing a CDP and paying a 10 ADA fee}
\label{cdp-close-figure}
\end{figure}}

\hypertarget{cdp-mint-iasset}{%
\subparagraph{CDP: Mint iAsset}\label{cdp-mint-iasset}}

Mints iAsset using an existing CDP

\begin{tabularx}{\linewidth}{l|l|L}
\toprule
\textbf{Type} & \textbf{Amount} & \textbf{Description}
\tabularnewline
\midrule
\endhead
\textbf{Redeemer} & N.A. & AdjustCDP
\tabularnewline
\midrule
\textbf{Consume} & 1 & CDP UTXO that represents the user's current
position
\tabularnewline
\midrule
\textbf{Reference} & 1 & iAsset UTXO that serves to identify the iAsset
the CDP is for
\tabularnewline
\midrule
\textbf{Reference} & 1 & UTXO containing the OracleAssetNFT with a datum
describing the iAsset price
\tabularnewline
\midrule
\textbf{Mint} & \(\infty\) & iAsset tokens the user selected to
mint
\tabularnewline
\midrule
\textbf{Output} & 1 & CDP UTXO that represents the user's adjusted
CDP
\tabularnewline
\midrule
\textbf{Output} & 1 & A new UTXO to the user wallet containing the newly
minted iAsset tokens
\tabularnewline
\bottomrule
\end{tabularx}

\hypertarget{cdp-mint-iasset-figure}{%
\begin{figure}[htbp]
\centering
\includesvg[width=17cm]{
images/endpoints/CDP/CDP-Mint-Asset.svg}
\caption{Example of using a CDP to mint 100 iAsset}
\label{cdp-mint-iasset-figure}
\end{figure}}

\hypertarget{cdp-burn-iasset}{%
\subparagraph{CDP: Burn iAsset}\label{cdp-burn-iasset}}

Burns iAsset using an existing CDP

\begin{tabularx}{\linewidth}{l|l|L}
\toprule
\textbf{Type} & \textbf{Amount} & \textbf{Description}
\tabularnewline
\midrule
\endhead
\textbf{Redeemer} & N.A. & AdjustCDP
\tabularnewline
\midrule
\textbf{Consume} & 1 & CDP UTXO that represents the user's current
position
\tabularnewline
\midrule
\textbf{Consume} & 1+ & UTXOs from the user's wallet containing the
iAsset tokens to be burned
\tabularnewline
\midrule
\textbf{Reference} & 1 & iAsset UTXO that serves to identify the iAsset
that the CDP is for
\tabularnewline
\midrule
\textbf{Burn} & \(\infty\) & The iAsset tokens the user requested to
burn
\tabularnewline
\midrule
\textbf{Output} & 1 & CDP UTXO that represents the user's adjusted
position
\tabularnewline
\midrule
\textbf{Output} & 1 & New UTXO to the user wallet returning change (if
any)
\tabularnewline
\bottomrule
\end{tabularx}

\hypertarget{cdp-burn-iasset-figure}{%
\begin{figure}[htbp]
\centering
\includesvg[width=17cm]{
images/endpoints/CDP/CDP-Mint-Asset.svg}
\caption{Example of using a CDP to burn 50 iAsset}
\label{cdp-burn-iasset-figure}
\end{figure}}

\hypertarget{cdp-freeze}{%
\subparagraph{CDP: Freeze}\label{cdp-freeze}}

Makes an existing CDP no longer interactable by its creator if it is
insolvent

\begin{tabularx}{\linewidth}{l|l|L}
\toprule
\textbf{Type} & \textbf{Amount} & \textbf{Description}
\tabularnewline
\midrule
\endhead
\textbf{Redeemer} & N.A. & FreezeCDP
\tabularnewline
\midrule
\textbf{Consume} & 1 & CDP UTXO that represents the user's current
position
\tabularnewline
\midrule
\textbf{Reference} & 1 & iAsset UTXO that serves to identify the iAsset
the CDP is for
\tabularnewline
\midrule
\textbf{Reference} & 1 & UTXO containing the OracleAssetNFT with a datum
describing the iAsset price
\tabularnewline
\midrule
\textbf{Output} & 1 & CDP UTXO that represents the frozen
CDP
\tabularnewline
\midrule
\textbf{Output} & 1 & New UTXO to the user wallet returning change (if
any)
\tabularnewline
\bottomrule
\end{tabularx}

\hypertarget{cdp-freeze-figure}{%
\begin{figure}[htbp]
\centering
\includesvg[width=17cm]{
images/endpoints/CDP/CDP-Freeze-CDP.svg}
\caption{Example of freezing a CDP, thereby removing the creator as
an owner}
\label{cdp-freeze-figure}
\end{figure}}

\hypertarget{cdp-liquidate}{%
\subparagraph{CDP: Liquidate}\label{cdp-liquidate}}

Withdraws ADA collateral from a CDP and transfers it to a SP if the CDP
is frozen

\begin{tabularx}{\linewidth}{l|l|L}
\toprule
\textbf{Type} & \textbf{Amount} & \textbf{Description}
\tabularnewline
\midrule
\endhead
\textbf{Redeemer} & N.A. & Liquidate
\tabularnewline
\midrule
\textbf{Redeemer} & N.A. & LiquidateCDP
\tabularnewline
\midrule
\textbf{Consume} & 1 & CDP UTXO that represents the frozen CDP to
liquidate
\tabularnewline
\midrule
\textbf{Consume} & 1 & SP UTXO that contains iAsset tokens to repay the
debt
\tabularnewline
\midrule
\textbf{Burn} & 0/1 & If all debt is repaid, then the CDPToken of the
frozen CDP is burned
\tabularnewline
\midrule
\textbf{Output} & 1 & SP UTXO that with the added collateral from the
frozen CDP
\tabularnewline
\bottomrule
\end{tabularx}

\hypertarget{cdp-liquidate-figure}{%
\begin{figure}[htbp]
\centering
\includesvg[width=17cm]{
images/endpoints/CDP/CDP-Liquidate-CDP.svg}
\caption{Example of a CDP with a debt of 50 iAsset and collateral
of 500 ADA being liquidated, with the collateral being transferred
to the Stability Pool, and the iAsset from the Stability Pool
being burned}
\label{cdp-liquidate-figure}
\end{figure}}

\hypertarget{cdp-merge}{%
\subparagraph{CDP: Merge}\label{cdp-merge}}

Closes one or more CDPs and transfers all CDP state into a single CDP

\begin{tabularx}{\linewidth}{l|l|L}
\toprule
\textbf{Type} & \textbf{Amount} & \textbf{Description}
\tabularnewline
\midrule
\endhead
\textbf{Redeemer} & N.A. & NergeCDPs
\tabularnewline
\midrule
\textbf{Redeemer} & N.A. & MergeAuxiliary, takes a CDP UTXO as a
parameter which identifies the main UTXO to keep and have others UTXOs
merged into
\tabularnewline
\midrule
\textbf{Consume} & 2+ & CDP UTXOs of the frozen CDPs
\tabularnewline
\midrule
\textbf{Output} & 1 & CDP UTXO representing all the frozen CDPs
combined
\tabularnewline
\bottomrule
\end{tabularx}

\hypertarget{cdp-merge-utxo-figure}{%
\begin{figure}[htbp]
\centering
\includesvg[width=17cm]{
images/endpoints/CDP/CDP-Merge-CDPs.svg}
\caption{Example of 5 CDPs being merged together}
\label{cdp-merge-utxo-figure}
\end{figure}}

\hypertarget{stability-pool}{%
\subsection{Stability Pool}\label{stability-pool}}

The SP contract is used as a pool of iAssets to be used for liquidation.
It is important to understand how the Snapshot works to understand how
the liquidations and account withdrawals work.

\begin{tabularx}{\linewidth}{l|L|L}
\caption{Stability Pool native tokens}
\tabularnewline
\toprule
\textbf{Name} & \textbf{Description} & \textbf{Minting
Policy}
\tabularnewline
\midrule
\endhead
\textbf{StabilityPoolToken} & Identify the authentic StabilityPool
output & The transaction must spend GovNFT
\tabularnewline
\midrule
\textbf{AccountToken} & Identify an authentic StabilityPoolAccount
output & The transaction must spend StabilityPoolToken
\tabularnewline
\bottomrule
\end{tabularx}

\hypertarget{stability-pool-parameters}{%
\subsubsection{Stability Pool
Parameters}\label{stability-pool-parameters}}

\begin{itemize}
\item
  \texttt{assetSymbol~::~CurrencySymbol}. The minting policy for
  iAssets.
\item
  \texttt{stabilityPoolToken~::~StabilityPoolToken}. The token
  identifying an authentic SP output.
\item
  \texttt{accountToken~::~StabilityPoolToken}. The token identifying an
  authentic SP Account output.
\item
  \texttt{cdpToken~::~CDPToken}. Token for identifying authentic CDP
  output.
\item
  \texttt{versionRecordToken~::~VersionRecordToken}. Token for
  identifying the version record for a protocol upgrade.
\item
  \texttt{collectorValHash~::~ValidatorHash}. The validator hash for the
  collector contract.
\item
  \texttt{govNFT~::~GovNFT}. NFT for identifying authentic governance
  parameters
\end{itemize}

\begin{tabularx}{\linewidth}{l|L|L|L}
\caption{Stability Pool outputs}
\tabularnewline
\toprule
\textbf{Type} & \textbf{Description} & \textbf{Datum} &
\textbf{Values}
\tabularnewline
\midrule
\endhead
\textbf{StabilityPool}
&
Each StabilityPool output holds iAssets to be used for
liquidations
&
\emph{spIAsset}: The name of the iAsset that this SP is for

\emph{spSnapshot}: The snapshot of funds for the SP. See Snapshot

\emph{epochToScaleToSum}: A map of the sum of funds for a particular
epoch and scale
&
\emph{StabilityPoolToken}: 1

\emph{iAsset}: Funded by stability providers

\emph{ADA}: Collateral transferred to SP from liquidated CDPs
\tabularnewline
\midrule
\textbf{EpochToScaleToSum}
&
Archives EpochToScaleToSum records
&
\emph{sessSnapshot}: A snapshot of EpochToScaleToSum

\emph{sessAsset}: The name of the iAsset that this snapshot is for
&
\emph{SnapshotToken}: 1
\tabularnewline
\midrule
\textbf{Account}
&
Each Account output holds a range of iAsset flavors
&
\emph{accOwner}: The owner of the SP Account

\emph{accIAsset}: The name of the iAsset that this SP Account is for

\emph{accSnapshot}: The snapshot of funds from the SP at the time of
deposit
&
\emph{AccountToken}: 1
\tabularnewline
\bottomrule
\end{tabularx}

\hypertarget{sp-endpoints}{%
\subsubsection{SP Endpoints}\label{sp-endpoints}}

\hypertarget{sp-create-account}{%
\subparagraph{SP: Create Account}\label{sp-create-account}}

Creates an account with a SP the first time a user deposits iAsset

\begin{tabularx}{\linewidth}{l|l|L}
\toprule
\textbf{Type} & \textbf{Amount} & \textbf{Description}
\tabularnewline
\midrule
\endhead
\textbf{Redeemer} & N.A. & CreateAccount, takes as a parameter a public
key hash corresponding to a user's wallet and an amount of iAsset to
stake
\tabularnewline
\midrule
\textbf{Consume} & 1 & SP UTXO representing the global state for the
iAsset type being deposited
\tabularnewline
\midrule
\textbf{Mint} & 1 & Account Token representing the user's SP
position
\tabularnewline
\midrule
\textbf{Output} & 1 & SP UTXO with the updated global
state
\tabularnewline
\midrule
\textbf{Output} & 1 & Account UTXO holding the user's Account
Token
\tabularnewline
\bottomrule
\end{tabularx}

\hypertarget{sp-create-figure}{%
\begin{figure}[htbp]
\centering
\includesvg[width=17cm]{
images/endpoints/Stability Pool/Create-Account.svg}
\caption{Example of a user making their first deposit into a
Stability Pool, depositing 50 iAsset, 2 ADA, and paying a 5 ADA
fee}
\label{sp-create-figure}
\end{figure}}

\hypertarget{sp-add-iasset}{%
\subparagraph{SP: Add iAsset}\label{sp-add-iasset}}

Adds more iAsset to user's SP account

\begin{tabularx}{\linewidth}{l|l|L}
\toprule
\textbf{Type} & \textbf{Amount} & \textbf{Description}
\tabularnewline
\midrule
\endhead
\textbf{Redeemer} & N.A. & AdjustAccount, takes as a parameter an amount
of iAsset
\tabularnewline
\midrule
\textbf{Redeemer} & N.A. & SpendAccount
\tabularnewline
\midrule
\textbf{Consume} & 1 & SP UTXO representing the global
state
\tabularnewline
\midrule
\textbf{Consume} & 1 & Account UTXO representing the user's Stability
Pool account to be adjusted
\tabularnewline
\midrule
\textbf{Consume} & 1+ & UTXOs containing the user's iAsset to be
deposited into their SP account
\tabularnewline
\midrule
\textbf{Output} & 1 & SP UTXO with the updated global
state
\tabularnewline
\midrule
\textbf{Output} & 1 & Account UTXO representing the user's updated SP
account
\tabularnewline
\bottomrule
\end{tabularx}

\hypertarget{sp-adjust-figure}{%
\begin{figure}[htbp]
\centering
\includesvg[width=17cm]{
images/endpoints/Stability Pool/Adjust-Account.svg}
\caption{Example of a user depositing 50 iAsset into the Stability
Pool, paying a 1 ADA fee}
\label{sp-adjust-figure}
\end{figure}}

\hypertarget{sp-close-account}{%
\subparagraph{SP: Close Account}\label{sp-close-account}}

Closes a user's SP account

\begin{tabularx}{\linewidth}{l|l|L}
\toprule
\textbf{Type} & \textbf{Amount} & \textbf{Description}
\tabularnewline
\midrule
\endhead
\textbf{Redeemer} & N.A. & Close
\tabularnewline
\midrule
\textbf{Redeemer} & N.A. & SpendAccount
\tabularnewline
\midrule
\textbf{Consume} & 1 & Stability Pool UTXO representing the global
state
\tabularnewline
\midrule
\textbf{Consume} & 1 & Account UTXO representing the user's Stability
Pool account to be closed
\tabularnewline
\midrule
\textbf{Burn} & 1 & Account Token representing the user's former
Stability Pool account
\tabularnewline
\midrule
\textbf{Output} & 1 & Stability Pool UTXO with the updated global
state
\tabularnewline
\midrule
\textbf{Output} & 1 & UTXO containing the iAsset that was deposited in
the Stability Pool account
\tabularnewline
\bottomrule
\end{tabularx}

\hypertarget{sp-close-figure}{%
\begin{figure}[htbp]
\centering
\includesvg[width=17cm]{
images/endpoints/Stability Pool/Close-Account.svg}
\caption{Example of a user withdrawing 100 iAsset from the
Stability Pool}
\label{sp-close-figure}
\end{figure}}

\hypertarget{stability-pool-recordepochtoscaletosum}{%
\subparagraph{Stability Pool:
RecordEpochToScaleToSum}\label{stability-pool-recordepochtoscaletosum}}

Archives EpochToScaleToSum records

\begin{tabularx}{\linewidth}{l|l|L}
\toprule
\textbf{Type} & \textbf{Amount} & \textbf{Description}
\tabularnewline
\midrule
\endhead
\textbf{Redeemer} & N.A. & RecordEpochToScaleToSum
\tabularnewline
\midrule
\textbf{Consume} & 1 & SP UTXO representing the global
state
\tabularnewline
\midrule
\textbf{Consume} & 1 & Account UTXO representing the user's Stability
Pool account to be closed
\tabularnewline
\midrule
\textbf{Burn} & 1 & Account Token representing the user's former SP
account
\tabularnewline
\midrule
\textbf{Output} & 1 & SP UTXO with the updated global
state
\tabularnewline
\midrule
\textbf{Output} & 1 & UTXO containing the iAsset that was deposited in
the SP account
\tabularnewline
\bottomrule
\end{tabularx}

\hypertarget{staking-1}{%
\subsection{Staking}\label{staking-1}}

The Staking contract is used primarily by the Indigo DAO Governance
package for proving the ownership of INDY tokens and locking those
tokens upon voting. The Staking contract also includes functionality to
collect protocol fees from Collector UTXOs.

\begin{tabularx}{\linewidth}{l|L|L}
\caption{Staking native tokens}
\tabularnewline
\toprule
\textbf{Name} & \textbf{Description} & \textbf{Minting
Policy}
\tabularnewline
\midrule
\endhead
\textbf{StakingManagerNFT}
&
The NFT identifies the authentic StakingManager output

The NFT must be stored in the StakingManager output

Validator scripts ensure that this NFT always stays at the
StakingManager output
&
The protocol mints exactly 1 token, before launch
\tabularnewline
\midrule
\textbf{StakingToken} & Identify the authentic StakingPosition output &
The transaction must consume a StakingManagerNFT or a
StakingToken
\tabularnewline
\bottomrule
\end{tabularx}

\hypertarget{parameters}{%
\subsubsection{Parameters}\label{parameters}}

\begin{itemize}
\item
  \texttt{stakingManagerNFT~::~StakingManagerNFT}. NFT of
  StakingManager.
\item
  \texttt{stakingToken~::~StakingToken}. Token for identifying authentic
  Staking Position output.
\item
  \texttt{indyToken~::~INDY}.
\item
  \texttt{pollToken~::~PollToken}. Token identifying authentic Poll
  output.
\item
  \texttt{versionRecordToken~::~VersionRecordToken}. Token identifying
  the VersionRegistry output.
\item
  \texttt{collectorValHash~::~ValidatorHash}. The collector script, used
  as a bridge between Staking and Poll Script.
\item
  \texttt{cdpToken~::~CDPToken}. Necessary for OffChain Endpoint to
  construct CollectorScriptParams.
\end{itemize}

\begin{tabularx}{\linewidth}{l|L|L|l}
\caption{Staking outputs}
\tabularnewline
\toprule
\textbf{Type} & \textbf{Description} & \textbf{Datum} &
\textbf{Values}
\tabularnewline
\midrule
\endhead
\textbf{StakingManager}
&
Only one output of this type is stored in the script

To create a StakingPosition output, the user must consume this output in
the transaction
&
\emph{totalStake}: The total amount of staked INDY

\emph{mSnapshot}: The snapshot of ADA rewards for INDY stakers
&
\emph{StakingManagerNFT}: 1
\tabularnewline
\midrule
\textbf{StakingPosition} & An individual user's INDY staking position &
\emph{owner}: The owner of the staking position \emph{lockedAmount}: A
map of Poll ID to (Vote Amount, Proposal End Time) \emph{pSnapshot}: The
snapshot of ADA rewards for the INDY staker & \emph{StakingToken}:
1
\tabularnewline
\bottomrule
\end{tabularx}

\hypertarget{staking-endpoints}{%
\subsubsection{Staking Endpoints}\label{staking-endpoints}}

\hypertarget{staking-create}{%
\subparagraph{Staking: Create}\label{staking-create}}

Creates a user's staking position

\begin{tabularx}{\linewidth}{l|l|L}
\toprule
\textbf{Type} & \textbf{Amount} & \textbf{Description}
\tabularnewline
\midrule
\endhead
\textbf{Redeemer} & N.A. & CreateStakingPosition
\tabularnewline
\midrule
\textbf{Consume} & 1 & Staking Manager UTXO representing the global
state of staking positions
\tabularnewline
\midrule
\textbf{Consume} & 1+ & UTXOs containing the user's INDY to be
staked
\tabularnewline
\midrule
\textbf{Mint} & 1 & Staking Token representing the user's staking
position
\tabularnewline
\midrule
\textbf{Output} & 1 & Staking Manager UTXO with the updated global
state
\tabularnewline
\midrule
\textbf{Output} & 1 & Staking Position UTXO holding the user's Staking
Token
\tabularnewline
\bottomrule
\end{tabularx}

\hypertarget{staking-create-figure}{%
\begin{figure}[htbp]
\centering
\includesvg[width=17cm]{
images/endpoints/Staking/Create-Staking-Position.svg}
\caption{Example of a user staking INDY for the first time,
depositing 50 INDY}
\label{staking-create-figure}
\end{figure}}

\hypertarget{staking-unstake}{%
\subparagraph{Staking: Unstake}\label{staking-unstake}}

Unstakes a user's staking position

\begin{tabularx}{\linewidth}{l|l|L}
\toprule
\textbf{Type} & \textbf{Amount} & \textbf{Description}
\tabularnewline
\midrule
\endhead
\textbf{Redeemer} & N.A. & UpdateTotalStake
\tabularnewline
\midrule
\textbf{Redeemer} & N.A. & Unstake
\tabularnewline
\midrule
\textbf{Consume} & 1 & Staking Manager UTXO representing the global
state of staking positions
\tabularnewline
\midrule
\textbf{Consume} & 1 & Staking Position UTXO representing the user's
staking position
\tabularnewline
\midrule
\textbf{Burn} & 1 & Staking Token representing the user's former staking
position
\tabularnewline
\midrule
\textbf{Output} & 1 & Staking Manager UTXO with the updated global
state
\tabularnewline
\midrule
\textbf{Output} & 1 & UTXOs containing the user's previously staked
INDY
\tabularnewline
\bottomrule
\end{tabularx}

\hypertarget{staking-unstake-figure}{%
\begin{figure}[htbp]
\centering
\includesvg[width=17cm]{
images/endpoints/Staking/Unstake.svg}
\caption{Example of a user unstaking 50 INDY}
\label{staking-unstake-figure}
\end{figure}}

\hypertarget{staking-stake}{%
\subparagraph{Staking: Stake}\label{staking-stake}}

Adds more INDY to a user's staking position

\begin{tabularx}{\linewidth}{l|l|L}
\toprule
\textbf{Type} & \textbf{Amount} & \textbf{Description}
\tabularnewline
\midrule
\endhead
\textbf{Redeemer} & N.A. & UpdateTotalStake
\tabularnewline
\midrule
\textbf{Redeemer} & N.A. & AdjustStakedAmount
\tabularnewline
\midrule
\textbf{Consume} & 1 & Staking Manager UTXO representing the global
state of staking positions
\tabularnewline
\midrule
\textbf{Consume} & 1 & Staking Position UTXO representing the user's
staking position
\tabularnewline
\midrule
\textbf{Consume} & 1+ & UTXOs containing the INDY to be
staked
\tabularnewline
\midrule
\textbf{Output} & 1 & Staking Manager UTXO with the updated global
state
\tabularnewline
\midrule
\textbf{Output} & 1 & Staking Position UTXO representing the user's
updated staking position
\tabularnewline
\bottomrule
\end{tabularx}

\hypertarget{staking-stake-figure}{%
\begin{figure}[htbp]
\centering
\includesvg[width=17cm]{
images/endpoints/Staking/Stake.svg}
\caption{Example of a user staking an additional 50 INDY}
\label{staking-stake-figure}
\end{figure}}

\hypertarget{staking-distribute}{%
\subparagraph{Staking: Distribute}\label{staking-distribute}}

Distributes fees from the Collector to the Staking Manager

\begin{tabularx}{\linewidth}{l|l|L}
\toprule
\textbf{Type} & \textbf{Amount} & \textbf{Description}
\tabularnewline
\midrule
\endhead
\textbf{Redeemer} & N.A. & Distribute
\tabularnewline
\midrule
\textbf{Redeemer} & N.A. & Collect
\tabularnewline
\midrule
\textbf{Consume} & 1 & Staking Manager UTXO representing the global
state of staking positions
\tabularnewline
\midrule
\textbf{Consume} & 1+ & Collector UTXOs containing the fees to
distribute
\tabularnewline
\midrule
\textbf{Output} & 1 & Staking Manager UTXO with the updated global
state
\tabularnewline
\bottomrule
\end{tabularx}

\hypertarget{staking-distribute-figure}{%
\begin{figure}[htbp]
\centering
\includesvg[width=17cm]{
images/endpoints/Staking/Distribute.svg}
\caption{Example of fees from the Collector being distributed to
the Staking Manager}
\label{staking-distribute-figure}
\end{figure}}

\hypertarget{staking-withdraw-rewards}{%
\subparagraph{Staking: Withdraw
Rewards}\label{staking-withdraw-rewards}}

Withdraw ADA rewards allocated to a user's staking position

\begin{tabularx}{\linewidth}{l|l|L}
\toprule
\textbf{Type} & \textbf{Amount} & \textbf{Description}
\tabularnewline
\midrule
\endhead
\textbf{Redeemer} & N.A. & UpdateTotalStake
\tabularnewline
\midrule
\textbf{Redeemer} & N.A. & AdjustStakedAmount
\tabularnewline
\midrule
\textbf{Consume} & 1 & Staking Manager UTXO representing the global
state of staking positions
\tabularnewline
\midrule
\textbf{Consume} & 1 & Staking Position UTXO representing the user's
staking position
\tabularnewline
\midrule
\textbf{Output} & 1 & Staking Manager UTXO with the updated global
state
\tabularnewline
\midrule
\textbf{Output} & 1 & Staking Position UTXO representing the user's
staking position
\tabularnewline
\bottomrule
\end{tabularx}

\filbreak

\hypertarget{staking-withdraw-rewards-figure}{%
\begin{figure}[htbp]
\centering
\includesvg[width=17cm]{
images/endpoints/Staking/Withdraw-Rewards.svg}
\caption{Example of a user withdrawing a 25 ADA reward}
\label{staking-withdraw-rewards-figure}
\end{figure}}

\hypertarget{staking-unlock}{%
\subparagraph{Staking: Unlock}\label{staking-unlock}}

Unlock staked INDY from a user's position to make those INDY
withdrawable

\begin{tabularx}{\linewidth}{l|l|L}
\toprule
\textbf{Type} & \textbf{Amount} & \textbf{Description}
\tabularnewline
\midrule
\endhead
\textbf{Redeemer} & N.A. & Unlock
\tabularnewline
\midrule
\textbf{Consume} & 1 & Staking Manager UTXO representing the global
state of staking positions
\tabularnewline
\midrule
\textbf{Consume} & 1 & Staking Position UTXO representing the user's
staking position
\tabularnewline
\midrule
\textbf{Output} & 1 & Staking Manager UTXO with the updated global
state
\tabularnewline
\midrule
\textbf{Output} & 1 & Staking Position UTXO representing the user's
staking position
\tabularnewline
\bottomrule
\end{tabularx}

\hypertarget{governance-1}{%
\subsection{Governance}\label{governance-1}}

Governance is a group of several contracts: Gov, Poll, Execute, and
VersionRegistry. The Gov contract stores protocol parameters and
controls the creation/ending of a Governance Poll. The Poll contract
handles the creation of vote shards, voting, merging of vote shards, AQB
calculations, and the ending of a Poll. The Execute contract takes the
result of a passed proposal and applies the appropriate action to the
contracts. The VersionRegistry contract handles the creation of Version
Records, which can be used by other protocol scripts to find an upgrade
path.

\begin{tabularx}{\linewidth}{l|L|L}
\caption{Governance native tokens}
\tabularnewline
\toprule
\textbf{Name} & \textbf{Description} & \textbf{Minting
Policy}
\tabularnewline
\midrule
\endhead
\textbf{GovNFT}
&
Identify authentic Governance output

Governance script ensures that this NFT always stays at the Governance
output
&
The protocol mints exactly 1 token at initialization
\tabularnewline
\midrule
\textbf{PollToken}
&
Identifies an authentic proposal

Validator scripts ensure that this token always stays at Poll
output
&
The transaction must consume GovNFT
\tabularnewline
\midrule
\textbf{UpgradeToken}
&
Identifies a passed proposal and the upgrade contract

Validator scripts ensure that this token always stays at Execute
output
&
The transaction must consume a PollToken
\tabularnewline
\midrule
\textbf{VersionRecordToken}
&
Identifies a potential upgrade path for a contract

Validator scripts ensure that this token always stays at VersionRegistry
output
&
The transaction must consume UpgradeToken
\tabularnewline
\bottomrule
\end{tabularx}

\hypertarget{execute-script-parameters}{%
\subsubsection{Execute Script
Parameters}\label{execute-script-parameters}}

\begin{itemize}
\item
  \texttt{govNFT~::~GovNFT}. NFT for identifying authentic Governance
  Script output.
\item
  \texttt{upgradeToken~::~UpgradeToken}. The asset class for identifying
  a valid upgrade token.
\item
  \texttt{iAssetToken~::~iAssetToken}. Token for identifying authentic
  iAsset output.
\item
  \texttt{stabilityPoolToken~::~StabilityPoolToken}. Token for
  identifying authentic SP output.
\item
  \texttt{versionRecordToken~::~VersionRecordToken}. Token for
  identifying the version record for a protocol upgrade.
\item
  \texttt{cdpValHash~::~ValidatorHash}. Hash of CDP script, used for
  verifying the output of a CDP.
\item
  \texttt{sPoolValHash~::~ValidatorHash}. Hash of SP script, used for
  verifying the output of a SP.
\item
  \texttt{versionRegistryValHash~::~ValidatorHash}. Hash of Version
  Registry script, used for verifying the output of a Version Registry.
\end{itemize}

\hypertarget{gov-script-parameters}{%
\subsubsection{Gov Script Parameters}\label{gov-script-parameters}}

\begin{itemize}
\item
  \texttt{govNFT~::~GovNFT}. NFT for identifying authentic Governance
  Script output.
\item
  \texttt{pollToken~::~PollToken}. The asset class for identifying a
  valid Poll token.
\item
  \texttt{upgradeToken~::~UpgradeToken}. The asset class for identifying
  a valid Upgrade token.
\item
  \texttt{indyAsset~::~INDY}.
\item
  \texttt{versionRecordToken~::~VersionRecordToken}. Token for
  identifying the version record for a protocol upgrade.
\item
  \texttt{pollManagerValHash~::~ValidatorHash}. Hash of Poll Manager
  script, used for verifying the output of a Poll.
\item
  \texttt{gBiasTime~::~POSIXTime}. Used to apply some leverage to the
  voting procedures.
\end{itemize}

\hypertarget{poll-manager-script-parameters}{%
\subsubsection{Poll Manager Script
Parameters}\label{poll-manager-script-parameters}}

\begin{itemize}
\item
  \texttt{govNFT~::~GovNFT}. NFT for identifying authentic Governance
  Script output.
\item
  \texttt{stakingManagerNFT~::~StakingManagerNFT}. NFT for identifying
  authentic Staking Manager output.
\item
  \texttt{pollToken~::~PollToken}. The asset class for identifying a
  valid Poll token.
\item
  \texttt{upgradeToken~::~UpgradeToken}. The asset class for identifying
  a valid Upgrade token.
\item
  \texttt{stakingToken~::~StakingToken}. The asset class for identifying
  a valid Staking Position token.
\item
  \texttt{indyAsset~::~INDY}.
\item
  \texttt{govExecuteValHash~::~ValidatorHash}. Hash of Execute script,
  used for verifying the output of a Upgrade token.
\item
  \texttt{stakingValHash~::~ValidatorHash}. Hash of Staking script, used
  for verifying the output of the Staking token.
\item
  \texttt{pBiasTime~::~POSIXTime}. Used to apply some leverage to the
  voting procedures.
\item
  \texttt{treasuryValHash~::~ValidatorHash}. Hash of the treasury
  script.
\item
  \texttt{initialIndyDistribution~::~Integer}. Used by the electorate
  calculation for the ITD value.
\item
  \texttt{totalINDYSupply~::~Integer}. Used by the electorate
  calculation for the t value.
\item
  \texttt{distributionSchedule~::~DistributionSchedule}. Used by the
  electorate calculation to map all the distributions and their intended
  distribution rates.
\item
  \texttt{shardsAddress~::~Address}. Poll shard validator address.
\end{itemize}

\hypertarget{poll-shard-script-parameters}{%
\subsubsection{Poll Shard Script
Parameters}\label{poll-shard-script-parameters}}

\begin{itemize}
\item
  \texttt{pollToken~::~PollToken}. The asset class for identifying a
  valid Poll token.
\item
  \texttt{stakingToken~::~StakingToken}. The asset class for identifying
  a valid Staking Position token.
\item
  \texttt{indyAsset~::~INDY}.
\item
  \texttt{stakingValHash~::~ValidatorHash}. Hash of Staking script, used
  for verifying the output of the Staking token.
\end{itemize}

\hypertarget{version-record-script-parameters}{%
\subsubsection{Version Record Script
Parameters}\label{version-record-script-parameters}}

\begin{itemize}
\item
  \texttt{upgradeToken~::~UpgradeToken}. The asset class for identifying
  a valid Upgrade token.
\end{itemize}

\begin{tabularx}{\linewidth}{l|L|L|l}
\caption{Governance outputs}
\tabularnewline
\toprule
\textbf{Type} & \textbf{Description} & \textbf{Datum} &
\textbf{Values}
\tabularnewline
\midrule
\endhead
\textbf{Governance}
&
Only one output of this type is stored in the script

To create a Poll output, the user must consume this output in the
transaction

To store the protocol parameters
&
\emph{currentProposal}: The number of opened proposals

\emph{protocolParams}: The parameters of the protocol

\emph{currentVersion}: The current version of the protocol, starting at
0

\emph{protocolStartTime}: The time that the protocol starts
&
\emph{GovNFT}: 1
\tabularnewline
\midrule
\textbf{Poll Manager}
&
The Poll Manager acts as a central UTXO that manages the content of the
poll
&
\emph{pId}: The identifying key for this particular proposal

\emph{pOwner}: The pub key hash of the owner of the poll

\emph{pContent}: The intended action of this poll: ProposeAsset,
MigrateAsset, ModifyProtocolParams, UpgradeProtocol, and TextProposal

\emph{pStatus}: The count of yes and no votes

\emph{pEndTime}: The time in which the poll should be ended

\emph{pCreatedShards}: The number of shards created

\emph{pTalliedShards}: The number of shards tallied and merged into Poll
Manager

\emph{pTotalShards}: The number of shards in total

\emph{pProposeEndTime}: The time in which all of the poll shards must be
created within

\emph{pExpirationTime}: The time in which the poll should expire

\emph{pProtocolVersion}: The protocol version at the time the poll UTXO
was created
&
\emph{PollToken}: 1
\tabularnewline
\midrule
\textbf{Poll Shard}
&
A derivation of the Poll Manager that stores some votes
&
\emph{psId}: The identifying key for this particular proposal

\emph{psStatus}: The count of yes and no votes

\emph{psEndTime}: The time in which the poll should be ended

\emph{psManagerAddress}: The address of the poll manager script
&
\emph{PollToken}: 1
\tabularnewline
\midrule
\textbf{Upgrade}
&
This output can be consumed to process a passed proposal
&
\emph{uId}: The identifying key for the passed proposal this upgrade was
derived from

\emph{uContent}: The intended action of this upgrade: ProposeAsset,
MigrateAsset, ModifyProtocolParams, UpgradeProtocol, and TextProposal

\emph{uPassedTime}: The time in which the poll was passed

\emph{uEndTime}: The time in which the upgrade should be deemed
"expired"

\emph{uProtocolVersion}: The protocol version at the time the upgrade
UTXO was created
&
\emph{UpgradeToken}: 1
\tabularnewline
\midrule
\textbf{VersionRecord}
&
Given a particular version id, the path for upgrading to a new
validator
&
\emph{versionId}: The version that the record is associated with.
Version starts at 0 at genesis and works up

\emph{versionPaths}: A map of the validator name that should be upgraded
and the currency symbol that can be used to process the upgrade
&
\emph{VersionRecordToken}: 1
\tabularnewline
\bottomrule
\end{tabularx}

\hypertarget{governance-endpoints}{%
\subsubsection{Governance Endpoints}\label{governance-endpoints}}

\hypertarget{governance-create-proposal}{%
\subparagraph{Governance: Create
Proposal}\label{governance-create-proposal}}

Creates a proposal to enact changes

\begin{tabularx}{\linewidth}{l|l|L}
\toprule
\textbf{Type} & \textbf{Amount} & \textbf{Description}
\tabularnewline
\midrule
\endhead
\textbf{Redeemer} & N.A. & CreatePoll, takes as parameters the time the
poll voting period should end, a public key hash corresponding to a
user's wallet, and the Poll's type (e.g.: ProposeAsset, MigrateAsset,
etc.)
\tabularnewline
\midrule
\textbf{Consume} & 1 & Governance UTXO
\tabularnewline
\midrule
\textbf{Consume} & 1+ & INDY to be deposited to create the
proposal
\tabularnewline
\midrule
\textbf{Mint} & 1 & Poll Token representing the newly created
proposal
\tabularnewline
\midrule
\textbf{Output} & 1 & Governance UTXO
\tabularnewline
\midrule
\textbf{Output} & 1 & Poll Manager UTXO that represents the
proposal
\tabularnewline
\bottomrule
\end{tabularx}

\hypertarget{governance-create-proposal-figure}{%
\begin{figure}[htbp]
\centering
\includesvg[width=17cm]{
images/endpoints/Governance/Governance-Create-Proposal.svg}
\caption{Example of a user depositing 50 INDY and 2 ADA to create
a proposal}
\label{governance-create-proposal-figure}
\end{figure}}

\hypertarget{governance-vote}{%
\subparagraph{Governance: Vote}\label{governance-vote}}

Vote on an open proposal

\begin{tabularx}{\linewidth}{l|l|L}
\toprule
\textbf{Type} & \textbf{Amount} & \textbf{Description}
\tabularnewline
\midrule
\endhead
\textbf{Redeemer} & N.A. & Vote, takes as a parameter the vote choice
(yes or no)
\tabularnewline
\midrule
\textbf{Redeemer} & N.A. & Lock
\tabularnewline
\midrule
\textbf{Consume} & 1 & Poll Shard UTXO to cast the vote
with
\tabularnewline
\midrule
\textbf{Consume} & 1 & Staking Position UTXO representing the user's
voting power
\tabularnewline
\midrule
\textbf{Output} & 1 & Poll Shard UTXO with the vote
recorded
\tabularnewline
\midrule
\textbf{Output} & 1 & Staking Position UTXO representing the user's
voting power
\tabularnewline
\bottomrule
\end{tabularx}

\hypertarget{governance-vote-figure}{%
\begin{figure}[htbp]
\centering
\includesvg[width=17cm]{
images/endpoints/Governance/Governance-Vote.svg}
\caption{Example of a user casting their vote}
\label{governance-vote-figure}
\end{figure}}

\hypertarget{governance-create-shards}{%
\subparagraph{Governance: Create
Shards}\label{governance-create-shards}}

Create one or more shards to allow users to vote on proposals

\begin{tabularx}{\linewidth}{l|l|L}
\toprule
\textbf{Type} & \textbf{Amount} & \textbf{Description}
\tabularnewline
\midrule
\endhead
\textbf{Redeemer} & N.A. & CreateShards, takes as a parameter the time
the poll voting period should end
\tabularnewline
\midrule
\textbf{Consume} & 1 & Poll Manager UTXO that represents the
proposal
\tabularnewline
\midrule
\textbf{Output} & \(\infty\) & Poll Shard UTXOs to record
votes
\tabularnewline
\bottomrule
\end{tabularx}

\hypertarget{governance-create-shards-figure}{%
\begin{figure}[htbp]
\centering
\includesvg[width=17cm]{
images/endpoints/Governance/Governance-Create-Shards.svg}
\caption{Example of a user creating three vote shards for their
proposal and depositing a refundable 6 ADA}
\label{governance-create-shards-figure}
\end{figure}}

\hypertarget{governance-merge-shards}{%
\subparagraph{Governance: Merge Shards}\label{governance-merge-shards}}

Merges one or more shards so that votes can be tallied

\begin{tabularx}{\linewidth}{l|l|L}
\toprule
\textbf{Type} & \textbf{Amount} & \textbf{Description}
\tabularnewline
\midrule
\endhead
\textbf{Redeemer} & N.A. & MergeShardsManager, takes as a parameter the
time the poll voting period should end
\tabularnewline
\midrule
\textbf{Redeemer} & N.A. & MergeShards
\tabularnewline
\midrule
\textbf{Consume} & 1 & Poll Manager UTXO that represents the
proposal
\tabularnewline
\midrule
\textbf{Consume} & \(\infty\) & Poll Shard UTXOs to merge
\tabularnewline
\midrule
\textbf{Output} & 1 & Poll Manager UTXO with the updated vote
count
\tabularnewline
\bottomrule
\end{tabularx}

\hypertarget{governance-merge-shards-figure}{%
\begin{figure}[htbp]
\centering
\includesvg[width=17cm]{
images/endpoints/Governance/Governance-Merge-Shards.svg}
\caption{Example of a user merging three vote shards for their
proposal and retrieving their original 6 ADA deposit}
\label{governance-merge-shards-figure}
\end{figure}}

\hypertarget{governance-end-proposal-passed}{%
\subparagraph{Governance: End Proposal
Passed}\label{governance-end-proposal-passed}}

End a proposal that has passed

\begin{tabularx}{\linewidth}{l|l|L}
\toprule
\textbf{Type} & \textbf{Amount} & \textbf{Description}
\tabularnewline
\midrule
\endhead
\textbf{Redeemer} & N.A. & EndPoll, takes as a parameter the time the
poll voting period should end
\tabularnewline
\midrule
\textbf{Consume} & 1 & Poll Manager UTXO that represents the
proposal
\tabularnewline
\midrule
\textbf{Reference} & 1 & Governance UTXO
\tabularnewline
\midrule
\textbf{Mint} & 1 & Upgrade Token
\tabularnewline
\midrule
\textbf{Burn} & 1 & Poll Token
\tabularnewline
\midrule
\textbf{Output} & 1 & Upgrade UTXO
\tabularnewline
\bottomrule
\end{tabularx}

\hypertarget{governance-end-proposal-passed-figure}{%
\begin{figure}[htbp]
\centering
\includesvg[width=17cm]{
images/endpoints/Governance/Governance-End-Proposal-Passed.svg}
\caption{Example of a user ending their proposal that passed and
retrieving their original 50 INDY deposit}
\label{governance-end-proposal-passed-figure}
\end{figure}}

\hypertarget{governance-end-proposal-failed-or-expired}{%
\subparagraph{Governance: End Proposal (Failed or
Expired)}\label{governance-end-proposal-failed-or-expired}}

End a proposal that has failed or expired

\begin{tabularx}{\linewidth}{l|l|L}
\toprule
\textbf{Type} & \textbf{Amount} & \textbf{Description}
\tabularnewline
\midrule
\endhead
\textbf{Redeemer} & N.A. & EndPoll, takes as a parameter the time the
poll voting period should end
\tabularnewline
\midrule
\textbf{Consume} & 1 & Poll Manager UTXO that represents the
proposal
\tabularnewline
\midrule
\textbf{Reference} & 1 & Governance UTXO
\tabularnewline
\midrule
\textbf{Burn} & 1 & Poll Token
\tabularnewline
\midrule
\textbf{Output} & 1 & Treasury UTXO containing the INDY deposited when
the proposal was created
\tabularnewline
\bottomrule
\end{tabularx}

\hypertarget{governance-end-proposal-failed-figure}{%
\begin{figure}[htbp]
\centering
\includesvg[width=17cm]{
images/endpoints/Governance/Governance-End-Proposal-Failed-Expired.svg}
\caption{Example of a user ending a failed or expired proposal and
sending the deposited 50 INDY to the Treasury}
\label{governance-end-proposal-failed-figure}
\end{figure}}

\hypertarget{governance-execute-text-proposal}{%
\subparagraph{Governance: Execute Text
Proposal}\label{governance-execute-text-proposal}}

Execute a passed proposal containing text adopted by the DAO

\begin{tabularx}{\linewidth}{l|l|L}
\toprule
\textbf{Type} & \textbf{Amount} & \textbf{Description}
\tabularnewline
\midrule
\endhead
\textbf{Redeemer} & N.A. & EndPoll, takes as a parameter the time the
poll voting period should end
\tabularnewline
\midrule
\textbf{Redeemer} & N.A. & Execute
\tabularnewline
\midrule
\textbf{Consume} & 1 & Upgrade UTXO containing the Upgrade Token for the
passed proposal
\tabularnewline
\midrule
\textbf{Burn} & 1 & Upgrade Token
\tabularnewline
\bottomrule
\end{tabularx}

\hypertarget{governance-execute-text-proposal-figure}{%
\begin{figure}[htbp]
\centering
\includesvg[width=17cm]{
images/endpoints/Governance/Governance-Execute-Text-Proposal.svg}
\caption{Example of a user executing their passed text proposal
and retrieving their original 2 ADA deposit}
\label{governance-execute-text-proposal-figure}
\end{figure}}

\hypertarget{governance-execute-propose-asset}{%
\subparagraph{Governance: Execute Propose
Asset}\label{governance-execute-propose-asset}}

Execute a passed proposal adopted by the DAO to whitelist a new iAsset

\begin{tabularx}{\linewidth}{l|l|L}
\toprule
\textbf{Type} & \textbf{Amount} & \textbf{Description}
\tabularnewline
\midrule
\endhead
\textbf{Redeemer} & N.A. & Execute
\tabularnewline
\midrule
\textbf{Consume} & 1 & Upgrade UTXO containing the Upgrade Token for the
passed proposal
\tabularnewline
\midrule
\textbf{Burn} & 1 & Upgrade Token
\tabularnewline
\midrule
\textbf{Mint} & 1 & iAsset Token
\tabularnewline
\midrule
\textbf{Mint} & 1 & Stability Pool Token
\tabularnewline
\midrule
\textbf{Output} & 1 & iAsset UTXO representing the new whitelisted
iAsset
\tabularnewline
\midrule
\textbf{Output} & 1 & Stability Pool UTXO representing the Stability
Pool for the new whitelisted iAsset
\tabularnewline
\bottomrule
\end{tabularx}

\hypertarget{governance-execute-propose-asset-figure}{%
\begin{figure}[htbp]
\centering
\includesvg[width=17cm]{
images/endpoints/Governance/Governance-Execute-Propose-Asset.svg}
\caption{Example of a user executing their passed whitelist iAsset
proposal, enabling a new iAsset within the protocol, creating a
new Stability Pool, and retrieving their original 2 ADA deposit}
\label{governance-execute-propose-asset-figure}
\end{figure}}

\hypertarget{governance-migrate-asset}{%
\subparagraph{Governance: Migrate
Asset}\label{governance-migrate-asset}}

Execute a passed proposal adopted by the DAO to update an existing
iAsset

\begin{tabularx}{\linewidth}{l|l|L}
\toprule
\textbf{Type} & \textbf{Amount} & \textbf{Description}
\tabularnewline
\midrule
\endhead
\textbf{Redeemer} & N.A. & Execute
\tabularnewline
\midrule
\textbf{Redeemer} & N.A. & UpgradeAsset
\tabularnewline
\midrule
\textbf{Consume} & 1 & Upgrade UTXO containing the Upgrade Token for the
passed proposal
\tabularnewline
\midrule
\textbf{Consume} & 1 & iAsset UTXO representing the iAsset to
update
\tabularnewline
\midrule
\textbf{Burn} & 1 & Upgrade Token
\tabularnewline
\midrule
\textbf{Output} & 1 & iAsset UTXO representing the updated
iAsset
\tabularnewline
\bottomrule
\end{tabularx}

\hypertarget{governance-execute-migrate-asset-figure}{%
\begin{figure}[htbp]
\centering
\includesvg[width=17cm]{
images/endpoints/Governance/Governance-Execute-Migrate-Asset.svg}
\caption{Example of a user executing their passed proposal to
update an iAsset, and retrieving their original 2 ADA deposit}
\label{governance-execute-migrate-asset-figure}
\end{figure}}

\hypertarget{liquidity}{%
\subsection{Liquidity}\label{liquidity}}

This contract is meant to be a store for the LP Tokens for tracking INDY
token rewards. Users will store their LP Tokens meant for reward
gathering in this contract. An off-chain mechanism will then be used to
calculate and confirm user rewards.

\begin{tabularx}{\linewidth}{l|L|L}
\caption{Liquidity outputs}
\tabularnewline
\toprule
\textbf{Type} & \textbf{Description} & \textbf{Datum}
\tabularnewline
\midrule
\endhead
\textbf{Liquidity} & This output acts as a store of LP Tokens &
\emph{owner}: The owner of the Liquidity Position
\tabularnewline
\bottomrule
\end{tabularx}

\hypertarget{liquidity-endpoints}{%
\subsubsection{Liquidity Endpoints}\label{liquidity-endpoints}}

\hypertarget{liquidity-create}{%
\subparagraph{Liquidity: Create}\label{liquidity-create}}

Creates a user's liquidity position

\begin{tabularx}{\linewidth}{l|l|L}
\toprule
\textbf{Type} & \textbf{Amount} & \textbf{Description}
\tabularnewline
\midrule
\endhead
\textbf{Consume} & 1+ & UTXOs containing the user's LP Tokens to be
staked
\tabularnewline
\midrule
\textbf{Output} & 1 & Liquidity UTXO representing the user's staked LP
Tokens
\tabularnewline
\bottomrule
\end{tabularx}

\hypertarget{liquidity-create-figure}{%
\begin{figure}[htbp]
\centering
\includesvg[width=17cm]{
images/endpoints/Liquidity/Liquidity-Create.svg}
\caption{Example of a user staking 500 LP tokens and depositing a
refundable 2 ADA}
\label{liquidity-create-figure}
\end{figure}}

\hypertarget{liquidity-stake}{%
\subparagraph{Liquidity: Stake}\label{liquidity-stake}}

Adds more LP Tokens to a user's liquidity position

\begin{tabularx}{\linewidth}{l|l|L}
\toprule
\textbf{Type} & \textbf{Amount} & \textbf{Description}
\tabularnewline
\midrule
\endhead
\textbf{Consume} & 1 & Liquidity UTXO representing the user's staked LP
Tokens
\tabularnewline
\midrule
\textbf{Consume} & 1+ & UTXOs containing the user's LP Tokens to be
staked
\tabularnewline
\midrule
\textbf{Output} & 1 & Liquidity UTXO representing the user's staked LP
Tokens
\tabularnewline
\bottomrule
\end{tabularx}

\hypertarget{liquidity-update-figure}{%
\begin{figure}[htbp]
\centering
\includesvg[width=17cm]{
images/endpoints/Liquidity/Liquidity-Update.svg}
\caption{Example of a user staking an additional 500 LP tokens}
\label{liquidity-update-figure}
\end{figure}}

\hypertarget{liquidity-unstake}{%
\subparagraph{Liquidity: Unstake}\label{liquidity-unstake}}

Unstakes a user's staked LP Tokens

\begin{tabularx}{\linewidth}{l|l|L}
\toprule
\textbf{Type} & \textbf{Amount} & \textbf{Description}
\tabularnewline
\midrule
\endhead
\textbf{Consume} & 1 & Liquidity UTXO representing the user's staked LP
Tokens
\tabularnewline
\midrule
\textbf{Output} & 1 & UTXO holding the user's unstaked LP
Tokens
\tabularnewline
\bottomrule
\end{tabularx}

\hypertarget{liquidity-close-figure}{%
\begin{figure}[htbp]
\centering
\includesvg[width=17cm]{
images/endpoints/Liquidity/Liquidity-Close.svg}
\caption{Example of a user unstaking 500 LP tokens and receiving
back their 2 ADA deposit}
\label{liquidity-close-figure}
\end{figure}}

\hypertarget{collector}{%
\subsection{Collector}\label{collector}}

The Collector contract is an intermediary contract between protocol fee
collection and distribution. The collection of funds can occur by
sending funds directly to the Collector, or consuming an existing
Collector and the output being more funds than were input. To distribute
the funds, the Staking Manager can consume a Collector UTXO and use it
to send funds to INDY stakers.

\hypertarget{parameters-1}{%
\subsubsection{Parameters}\label{parameters-1}}

\begin{itemize}
\item
  \texttt{stakingManagerNFT~::~StakingManagerNFT}. NFT of
  StakingManager.
\item
  \texttt{stakingToken~::~StakingToken}. Token for identifying authentic
  Staking Position output.
\item
  \texttt{versionRecordToken~::~VersionRecordToken}. Token identifying
  the VersionRegistry output
\end{itemize}

\hypertarget{collector-endpoints}{%
\subsubsection{Collector Endpoints}\label{collector-endpoints}}

\hypertarget{collector-collect}{%
\subparagraph{Collector: Collect}\label{collector-collect}}

Collect Protocol Fees upon withdrawing a CDP's collateral or closing a
CDP

\begin{tabularx}{\linewidth}{l|l|L}
\toprule
\textbf{Type} & \textbf{Amount} & \textbf{Description}
\tabularnewline
\midrule
\endhead
\textbf{Redeemer} & N.A. & Collect
\tabularnewline
\midrule
\textbf{Consume} & 1 & Collector UTXO which may already contain
previously collected protocol fees
\tabularnewline
\midrule
\textbf{Consume} & 1 & CDP UTXO that represents a user's
CDP
\tabularnewline
\midrule
\textbf{Output} & 1 & Collector UTXO updated with the collected
fee
\tabularnewline
\bottomrule
\end{tabularx}

\hypertarget{treasury}{%
\subsection{Treasury}\label{treasury}}

The purpose of this contract is to hold the DAO Treasury funds. The DAO
Treasury will contain INDY that's vested over time according to Indigo's
tokenomics model. The funds in the DAO Treasury are intended to be spent
to help further develop, maintain and enhance the Indigo Protocol for
the betterment of its users and INDY holders. Until a future protocol
version upgrade allows those funds to be spent, the funds will be locked
in this contract.

\hypertarget{parameters-2}{%
\subsubsection{Parameters}\label{parameters-2}}

\begin{itemize}
\item
  \texttt{versionRecordToken~::~VersionRecordToken}. Token identifying
  the VersionRegistry output.
\end{itemize}

\begin{tabularx}{\linewidth}{l|L|L}
\caption{Treasury outputs}
\tabularnewline
\toprule
\textbf{Type} & \textbf{Description} & \textbf{Values}
\tabularnewline
\midrule
\endhead
\textbf{Treasury}
&
This output stores the DAO Treasury tokens
&
\emph{INDY}: The INDY stored in the Treasury

\emph{IdentityToken}: 1
\tabularnewline
\bottomrule
\end{tabularx}

\hypertarget{known-protocol-limitations}{%
\section{Known Protocol Limitations}\label{known-protocol-limitations}}

Indigo Protocol v1.0 was built to support a mainnet launch. However,
there are a few areas that could have additional optimizations and/or
fixes to be more scalable and to support a growing user base.

\hypertarget{stability-pool-contention}{%
\subsection{Stability Pool Contention}\label{stability-pool-contention}}

As described in the
\protect\hyperlink{stability-pool-liquidation-rewards}{SP Rewards
section}, SPs have two associated UTXOs that can lead to contention: SP
state, and account record. There exists one SP state UTXO per iAsset. SP
state must be updated upon the following actions:

\begin{enumerate}
\item
  Create an account record (i.e., deposit iAsset into the SP)
\item
  Adjust an account record (i.e., deposit iAsset into, or withdraw
  iAsset from, the SP)
\item
  Liquidate a CDP (i.e., withdraw iAsset from the SP)
\end{enumerate}

Updating SP state causes contention because only one update can be made
per SP per block. As a mitigation effort, multiple CDPs can be merged
into one to reduce the number of CDP liquidation actions required
against a single SP.

Contention still exists for users depositing and withdrawing iAssets
from a SP. To mitigate this effect, a
\protect\hyperlink{stability-pool-staking-fees}{fee mechanism} has been
implemented to disincentivize SP stakers from updating their positions
too frequently. When users create or adjust an account record, they will
be required to deposit a mimimum amount of ADA, which will be
redistributed to all SP stakers.

Additionally, contention is experienced when SP stakers withdraw their
owed rewards. Only one user can withdraw ADA rewards earned from
liquidations per SP per block.

\hypertarget{governance-contention}{%
\subsection{Governance Contention}\label{governance-contention}}

Users can deposit their INDY into the protocol to become
\protect\hyperlink{staking}{Members} and gain access to privileges such
as voting rights and reward collection. The
\protect\hyperlink{staking}{Staking Manager} UTXO is responsible for
managing staking positions of users. The Staking Manager must be updated
upon the following actions:

\begin{enumerate}
\item
  Create a staking position (i.e., deposit INDY into governance)
\item
  Adjust a staking position (i.e., deposit INDY into or withdraw INDY
  from governance)
\item
  Deposit staking reward (i.e., collect an ADA protocol fee)
\item
  Withdraw staking reward (i.e., redeem ADA reward for staking)
\end{enumerate}

Interacting with the Staking Manager causes contention because only one
update can be made per block. As a mitigation effort, the
\protect\hyperlink{collector}{Collector} can bundle staking rewards
collected by the protocol to reduce the number of staking reward
transactions deposited into the Staking Manager. Contention still exists
for INDY stakers depositing or withdrawing INDY or withdrawing ADA
rewards.

Additionally, contention exists for recording governance votes. To
improve scalability, votes are recorded using
\protect\hyperlink{governance-sharding}{individual shards}. Users can
pick unused shards to record their votes. While, theoretically, an
unlimited number of shards can be configured, the Cardano blockchain is
limited in the number of shards that can record votes per block. If
there are insufficient available shards, then users will have to wait
for a shard to become available before voting.

Shard collision can occur when two or more users select the same shard
to vote with; only one user will succeed with recording the vote, the
other users using the same shard will experience transaction errors. If
a user explicitly checks for shard availability before submitting a
transaction, another user may also select that same shard before the
transaction is processed in a block, thereby possibly resulting in
collision and transaction failure for either user.

\hypertarget{definitions-for-mathematical-notations}{%
\section{Definitions for Mathematical
Notations}\label{definitions-for-mathematical-notations}}

Throughout this document, references are made to mathematical equations.
Below is a summary of notations that may be used and their associated
meanings.

\hypertarget{sets}{%
\subsection{Sets}\label{sets}}

Values enclosed in \(\left\{ \  \right\}\) are a unique assortment of
values. Each value is separated by a comma (\(,\)).

\(\left\{ 10,20,30,40,50 \right\}\) means five values incrementing in
tens, beginning at 10 and ending at 50.

\hypertarget{summation}{%
\subsection{Summation}\label{summation}}

The \(\sum_{}^{}\ \)represents a sum of values. It can either be in the
form of \(\sum_{}^{}x\) or \(\sum_{i = 1}^{n}i\).

\(\sum_{}^{}x\) means to sum all values of a set. If \(x\) is a set of
\(\left\{ 1,2,3 \right\}\), then:

\[\sum_{}^{}x = 1 + 2 + 3 = 6\]

\(\sum_{i = 1}^{n}i\) means to iterate \(n\) times and sum the result of
\(x\). \(i\) begins at \(0\) increments until \(i\) equals \(n\). If
\(n\) is 3, then:

\[\sum_{i = 1}^{n}i = 1 + 2 + 3 = 6\]

\hypertarget{length-of-sets}{%
\subsection{Length of Sets}\label{length-of-sets}}

A set enclosed within \(\left| \  \right|\) represents the length of the
set.

\(\left| x \right|\) means the length of set \(x\). If \(x\) is
\(\left\{ 5,10,15 \right\}\), then \(\left| x \right| = 3\) because it
contains 3 elements in the set.

\hypertarget{indexes}{%
\subsection{Indexes}\label{indexes}}

A subscript (\(x_{i}\)) represents an associated variable or a value
within a set.

If \(x\) is a set and \(i\) is a number, then \(x_{i}\) means the
\(i\)\textsuperscript{th} element of the set \(x\). If \(x\) is
\(\left\{ 3,6,9 \right\}\), then \(x_{1}\) is 3, \(x_{2}\) is 6, and
\(x_{3}\) is 9. Thus, if \(i\) is 2, then \(x_{i}\) is 6 because it's
the 2\textsuperscript{nd} element of \(x\).

\hypertarget{mean-of-sets}{%
\subsection{Mean of Sets}\label{mean-of-sets}}

A set with \(\overline{}\) (a bar) above it represents the mean
(average) of the set.

\(\overline{x}\) means the mean of set \(x\), which is the sum of all
elements in the set divided by the length of the set. Alternatively,
\(\overline{x}\) can be expressed as:

\[\frac{\sum_{i = 1}^{\left| x \right|}x_{i}}{\left| x \right|}\]

If \(x\) is \(\left\{ 10,30,20,40 \right\}\), then:

\[\overline{x} = \frac{10 + 30 + 20 + 40}{4} = 25\]

\hypertarget{rounding}{%
\subsection{Rounding}\label{rounding}}

Values enclosed in \(\left\lceil \  \right\rceil\) or
\(\left\lfloor \  \right\rfloor\) represent the value either rounded up
or down to the nearest whole number.

\(\left\lceil x \right\rceil\) means ``ceil,'' or to round up to the
nearest whole number. If \(x\) is 0.5, then:
\(\left\lceil x \right\rceil = 1\).

\(\left\lfloor x \right\rfloor\) means ``floor,'' or to round down to
the nearest whole number. If \(x\) is 0.5, then:
\(\left\lfloor x \right\rfloor = 0\).

\hypertarget{scoped-variables}{%
\subsection{Scoped Variables}\label{scoped-variables}}

Sometimes equations may be simplified and made more readable using
scoped variables.

\(x = \begin{pmatrix}
\text{let\ }y\text{\ equal\ }1 \\
y \\
\end{pmatrix}\) means to create a variable called \(y\) with a value of
1, which can then be referenced throughout any component within the
\(\left( \  \right)\) it's defined within. Therefore, \(x\) is 1 because
the bottom-most statement is \(y\) and \(y\) is 1.

\hypertarget{conditional-statements}{%
\subsection{Conditional Statements}\label{conditional-statements}}

A statement proceeding \(\left\{ \  \right.\ \ \)without an
enclosing\(\left. \ \ \right\}\) is conditional. Conditional statements
take the form of \(\left\{ \begin{matrix}
x & \text{if\ }a > 0 \\
y & \text{otherwise} \\
\end{matrix} \right.\ \ \). They can have two or more conditions, such
as: \(\left\{ \begin{matrix}
x & \text{if\ }a > 0\text{\ and\ }a < 1 \\
y & \text{if\ }a > 50 \\
z & \text{otherwise} \\
\end{matrix} \right.\ \ \).

\(\left\{ \begin{matrix}
x & \text{if\ }a > 0\text{\ and\ }a < 1 \\
y & \text{if\ }a \geq 1 \\
z & \text{otherwise} \\
\end{matrix} \right.\ \ \) means that the value is determined by the
truthfulness of three conditions. If \(a\) is between 0 and 1, then the
statement is \(x\). If \(a\) is 1 or larger, then the statement is
\(y\). The only other possibility is \(a\) is 0 or smaller, in which
case the statement is \(z\).

``otherwise'' means if no other condition matches.

``and'' means that both conditions must be true.

``or'' means that either condition must be true.

\hypertarget{functions}{%
\subsection{Functions}\label{functions}}

Statements proceeding \(f:\left( \  \right) \mapsto\) represent a
callable function that can be referenced.

\(f:\left( a,b \right) \mapsto a + b\) means that \(f\) takes two values
and adds them together to determine the value. A reference to
\(f\left( 1,2 \right)\) equates to 3.

\hypertarget{minimum-and-maximums}{%
\subsection{Minimum and Maximums}\label{minimum-and-maximums}}

Minimum and maximum values within sets can be referenced using
\(min\left\{ \  \right\}\) and \(max\left\{ \  \right\}\) respectively.

\(min\left\{ 100,10,1000 \right\}\) means the lowest value out of the
set \(\left\{ 100,10,1000 \right\}\), therefore: 10.

\(max\left\{ 100,10,1000 \right\}\) means the highest value out of the
set \(\left\{ 100,10,1000 \right\}\), therefore: 1000.

\filbreak

\hypertarget{minimum-ada-to-create-utxo}{%
\section{Minimum ADA to Create UTXO}\label{minimum-ada-to-create-utxo}}

To create a UTXO on Cardano, a minimum amount of ADA is required to be
locked into the transaction. The amount of ADA deposit required to
create a UTXO is calculated using the formula:

\[x = ab + 160b\]

Where:

\begin{itemize}
\item
  \(x\) is the amount of ADA required to create a shard
\item
  \(a\) is the size of the UTXO of the transaction
\item
  \(b\) is the \emph{coinsPerUTxOByte} parameter of the Cardano
  blockchain\footnote{Calculating required ADA for UTXOs is described by
    the UTXO inference rules on page 16 of the
    \href{https://hydra.iohk.io/build/17586760/download/1/babbage-changes.pdf}{formal
    Cardano specification}.}
\end{itemize}

Upon closing the UTXO, the deposited ADA can be unlocked.

\end{sloppypar}
\end{document}
